\chapter{Introduction\label{cha:introduction}}

In the last few years we have seen increased interest in \emph{experimental 
mathematics} where computers are used to perform experiments and analyze data 
in greater quantities than humans are able to process by hand. Combinatorics is 
one field of mathematics that has greatly benefited from the advent of 
computers. Researchers routinely write computer programs that can quickly 
verify or refute hypotheses made by themselves to enable them to more
efficiently allocate their time and energy with regards to the direction of the 
research. 

Even more interesting are computer programs that can act as a source of
inspiration by finding obscure and hidden patterns in large problem spaces. 
This can help the human researcher by giving them the correct idea to further
pursue. One such tool is \Struct\ by \textcite{bean_automatic_2019} which can
automatically discover the structural rules of \emph{permutation 
classes}\footnote{We will define this and other terminology in 
Section~\ref{mesh tilings background}.} and thus conjecture the 
\emph{enumeration} of the set as a function of the length of the 
\emph{permutations} in the set. This is a problem whose solution can be hard to 
discover but easy to prove once found.

Other more complex computer programs exists but they often come with a drawback
or compromise between usability and effectiveness. \CombSpecSearcher\ by 
\textcite{bean_finding_2018} can find the enumeration of permutation classes, 
along with a proof of its validity, but it required significant work on the 
behalf of the researcher in implementing \emph{strategies} specific to the 
domain of permutations. Applying it to another domain would require similar
effort.  

\Struct\ and our extended version \CombCov\ have a lower barrier of entry for 
users in the hope that it will be more widely adopted by researchers. The lack 
of computer generated proofs are mitigated by the fact that the conjectures are 
often easily verified or disproved by the user.

% Researchers in the field of combinatorics often deal with problems that revolve around counting things. An example of this is computing the size of sets that are induced in some way by a combinatorial object as a function of its size $n$. Before the advent of computers this was a laborious task consisting of calculating the first few terms by hand and then trying to find the underlying formula which fit to these numbers and hopefully generalize to the entire sequence.

% In the last few years we have seen increased interest in \emph{experimental mathematics} where computers are used to perform experiments, analyze data and conjecture hypotheses in ways that guide, verify or disprove the human mathematician. One such tool is \Struct\ by \textcite{bean_automatic_2019} which can automatically discover the structural rules of \emph{permutation classes}\footnote{We will define this and other terminology in Section~\ref{sec:background}.} and thus conjecture the enumeration as a function of the length of permutations.

% We note that there also exist tools such as \CombSpecSearcher\ by \textcite{bean_finding_2018} that can find the enumeration of these sets, along with a proof of its validity, but they often require more work on the behalf of the researcher. In the case of \CombSpecSearcher\ the user needs to implement \emph{strategies} specific to their domain. Both \Struct\ and our extended version \CombCov\ have a lower barrier of entry for users in the hope that it will be more widely adopted by researchers. The lack of computer generated proofs are mitigated by the fact that the conjectures are often easily verified or disproved by the user.


%%%%%%%%%%%%%%%%%%%%%%%%%%%%%%%%%%%%%%%%%%%%%%%%%%%%%%%%%%%%%%%%%%%%%%%%%%%%%%%%
\section{Layout of thesis}

We dedicate Chapter~\ref{words chapter} to explaining how \CombCov\ works from a 
high-level perspective. In Section~\ref{words background} we define \emph{words}
as an intuitive example of \emph{avoidance sets} and in Section~\ref{words 
implementation} we explain how we implement them with \CombCov. In 
Section~\ref{words results} we present some results and analyze the 
effectiveness of \CombCov\ as a tool.

Chapter~\ref{mesh tilings chapter} concerns itself with our main application of 
\CombCov. In Section~\ref{mesh tilings background} we define \emph{permutations}
and \emph{permutation classes}, the domain in which \Struct\ operates, and 
\emph{mesh patterns} and \emph{mesh tilings}. Our implementation of mesh tilings
with  the framework is discussed in Section~\ref{mesh tilings implementation} 
and the results are presented in Section~\ref{mesh tilings results}.
