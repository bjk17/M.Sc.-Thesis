\chapter{Mesh tilings\label{mesh tilings chapter}}

In this chapter we review the terminology on permutations and mesh patterns 
before defining \emph{mesh tilings}. In Section~\ref{mesh tilings 
implementation} we explain how we apply \CombCov\ to mesh tilings and discuss 
the results in Section~\ref{mesh tilings results}. We consider this to be our
main contribution apart from the framework itself.

%%%%%%%%%%%%%%%%%%%%%%%%%%%%%%%%%%%%%%%%%%%%%%%%%%%%%%%%%%%%%%%%%%%%%%%%%%%%%%%%
\section{Background\label{mesh tilings background}}

\begin{definition}
  A \emph{(classical) permutation of length $n$} is a bijection from the set of 
  the first $n$ integers, $\llbracket n \rrbracket = \left\{ 1, 2, \ldots, n 
  \right\}$ to itself. The set of all permutations of length $n$ is denoted with 
  $\S_n$ and $\S = \bigcup\limits_{n=0}^{\infty}\S_n$ is the set of all 
  permutations.
\end{definition}

A permutation can also be viewed as a string $\pi(1)\cdots\pi(n)$ by listing the 
values that $\pi$ takes at each index. An example of this is $\pi = 35142$, a 
permutation of length five with $\pi(1) = 3$, $\pi(2) = 5$, $\pi(3) = 1$, 
$\pi(4) = 4$ and $\pi(5) = 2$. The visual \emph{grid representation} of $\pi$, 
denoted with $\Gr{\pi}$, is the plot of $\left\{ (i, \pi(i)) \mid i \in 
\llbracket n \rrbracket \right\}$ in a Cartesian coordinate system, shown in 
Figure~\ref{figure: Gr(35142)}. The only permutation of length 0 is called the 
\emph{empty permutation} and it is denoted by $\epsilon$.

\begin{figure}[htbp]
  \center
  \begin{tikzpicture}[scale=\pattdispscale, baseline={([yshift=-3pt]current bounding box.center)}]
    \def \n {5}
    \foreach \x in {1,...,\n} {
      \draw[gray] (0,\x) -- (\n+1,\x);
      \draw[gray] (\x,0) -- (\x,\n+1);
    }
    \foreach \x in {(1,3),(2,5),(3,1),(4,4),(5,2)} {\fill[black] \x circle (5pt);}
  \end{tikzpicture}
  \caption{The grid representation of the permutation $35142$}
  \label{figure: Gr(35142)}
\end{figure}

\begin{definition}
  The \emph{standardization} of a length $n$ string of unique numbers $s_1 
  \cdots s_n$ is the permutation $\sigma \in \S_n$ which has the same relative 
  ordering as the string, meaning that for every $i, j$, $\sigma(i) < \sigma(j)$ 
  if $s_i < s_j$. We write $\st(s_1 \cdots s_n) = \sigma$.
\end{definition}

Every \emph{index set} $\left\{ i_1, \ldots, i_k \right\} \subseteq \llbracket n 
\rrbracket$ (with $1 \leq i_1 < i_2 < \cdots < i_k \leq n$) \emph{induces} a 
substring $\pi(i_1)\cdots\pi(i_k)$ of length $k$ of $\pi$ and is called an 
\emph{occurrence} of the \emph{pattern} $p = \st(\pi(i_1)\cdots\pi(i_k))$ in 
$\pi$. We say that $\pi$ \emph{contains} $p$ as a \emph{subpermutation} or 
\emph{(classical) pattern} and use the notation $p \preceq \pi$. If there exists 
no index set that induces an occurrence of $p$ we say that $\pi$ \emph{avoids} 
$p$. The set of all patterns contained in $\pi$ is denoted with $\Delta(\pi) = 
\left\{ p \in \S \mid p \preceq \pi \right\}$ and $\Delta(\Pi) = \bigcup_{\pi 
\in \Pi}{\Delta(\pi)}$ for a set of permutations $\Pi$.

% Sometimes we abuse the notation and say that the subset $\left\{ \left( i_j, 
% \pi_j \right) \mid j = 1, \ldots, k \right\}$ of $\Gr{\pi}$ is the occurrence 
% of the pattern $p$ in $\pi$. 

\begin{example}
  The permutation $\pi = 35142$ contains the pattern $q = 213$, highlighted with 
  red circles in Figure~\ref{figure:q in pi}, because $\left\{ 1, 3, 4 \right\}$ 
  induces the substring $\pi(1)\pi(3)\pi(4)= 314$ that has the same relative 
  ordering as $q$ (meaning $\st(314) = 213 = q$). Note that this is the only 
  occurrence of $q$ in $\pi$. 
  \label{example:q in pi}

  \begin{figure}[htbp]
    \center
    \begin{tikzpicture}[scale=\pattdispscale, baseline={([yshift=-3pt]current bounding box.center)}]
      \def \n {5}
      \foreach \x in {1,...,\n} {
        \draw[gray] (0,\x) -- (\n+1,\x);
        \draw[gray] (\x,0) -- (\x,\n+1);
      }
      \foreach \x in {(1,3),(2,5),(3,1),(4,4),(5,2)} {\fill[black] \x circle (5pt);}
      \foreach \x in {(1,3),(3,1),(4,4)} {\draw[thick,red] \x circle (7pt);}
    \end{tikzpicture}
    \caption{An occurrence of $213$ in $35142$ highlighted in red}
    \label{figure:q in pi}
  \end{figure}
\end{example}

The set of all permutations of length $n$ that avoid $\pi$ is denoted by 
$\Avn{\pi}$ and we write $\Av{\pi} = \bigcup\limits_{n=0}^{\infty}\Avn{\pi}$. 
If $\Pi$ is a set of permutations we similarly write $\Av{\Pi} = 
\bigcap\limits_{\pi \in \Pi}\Av{\pi}$, i.e., avoiding a set of permutations 
means avoiding every permutation in the set. Note that $\Av{\emptyset} = \S$.

In contrast to the set of avoiding permutations, we define the 
\emph{containment set} as the complementary, i.e., for a set of permutations 
$\Pi$ we define $\Con{\Pi} = \S_n \setminus \Avn{\pi}$ and write $\Co{\Pi} = 
\bigcup\limits_{n=0}^{\infty}\Con{\Pi}$.

\begin{example}
  The set of all permutations avoiding $21$ is the set of \emph{identity} 
  permutations: \[ \Av{21} = \left\{ \epsilon, 1, 12, 123, \ldots \right\} \]
\end{example}

A \emph{permutation class} is a set $\mc{C}$ of permutations such that 
$\Delta(\mc{C}) \subseteq \mc{C}$. It it well known that $\Av{\Pi}$ is a 
permutation class for any set of permutations $\Pi$. A permutation class is 
\emph{``closed downwards''} as every pattern contained in every permutation in 
$\mc{C}$ is also in $\mc{C}$.


%%%%%%%%%%%%%%%%%%%%%%%%%%%%%%%%%%%%%%%%%%%%%%%%%%%%%%%%%%%%%%%%%%%%%%%%%%%%%%%%
\subsection{Mesh patterns\label{mesh patterns background}}

The concept of \emph{mesh patterns} in permutations was introduced by 
\textcite{branden_mesh_2011} as a pair
\[p = (\pi, R) \text{ with } \pi \in \S_k \text{ and } R \subseteq \llbracket 
0, k \rrbracket \times \llbracket 0, k \rrbracket \] where $\llbracket 0, k 
\rrbracket = \{0, 1, \ldots, k\}$ and $R$ is a set of Cartesian coordinates 
$\boks{i, j}$ denoting the lower left corners of the squares in the grid 
representation of $\pi$ which are \emph{shaded}. An example with $p = (\pi, R)$ 
where $\pi = 213$ and $R = \left\{ \boks{1,2}, \boks{1,3}, \boks{2,2} \right\}$ 
is shown in Figure~\ref{figure: Gr(213, shading)}.

\begin{figure}[htbp]
  \center
  \mpattern{scale=\patttablescale}{ 3 }{ 1/2, 2/1, 3/3 }{ 1/2, 1/3, 2/2 }
  \caption{The mesh pattern $p = \left(213, \left\{ \boks{1,2}, \boks{1,3}, \boks{2,2} \right\} \right)$}
  \label{figure: Gr(213, shading)}
\end{figure}

\begin{definition}[\citeauthor{branden_mesh_2011}]
  An occurrence of $p$ in $\sigma \in \S_n$ is a subset $\omega$ of 
  $\Gr{\sigma}$ such that there are order-preserving injections $\alpha, \beta 
  \colon \llbracket k \rrbracket \mapsto \llbracket n \rrbracket$ satisfying the 
  following two conditions. Firstly, $\omega$ is an occurrence of $\pi$ in the 
  classical sense, that is

    i. $\omega = \left\{ \left( \alpha(i), \beta(j) \right) \mid (i, j) \in \Gr{\pi} \right\}$

  \noindent
  Define $R_{ij} = \left[ \left( \alpha(i) + 1, \alpha(i + 1) - 1 \right) 
  \right] \times \left[ \left( \beta(j) + 1, \beta(j + 1) - 1 \right) \right]$ 
  or $i,j \in \llbracket 0, k \rrbracket$ where $\alpha(0) = \beta(0) = 0$ and 
  $\alpha(k + 1) = \beta(k + 1) = n + 1$. Then the second condition is

    ii. if $\boks{i, j} \in R$ then $R_{ij} \cap \Gr{\sigma} = \emptyset$.

  \noindent
  We call $R_{ij}$ the \emph{region corresponding to} $\boks{i,j}$ and $\pi$ 
  the \emph{underlying classical pattern} of $p$.
\end{definition}

Note that we defined an occurrence of a pattern in a classical permutation to be 
the subsequence of the permutation matching the pattern, but Brändén and 
Claesson defined an occurrence of a pattern in a mesh pattern to be the subset 
of the grid representation matching the pattern. For the sake of simplicity in 
what is to come we will keep with the original author's definition of 
occurrences.

\begin{example}
  Recall Example~\ref{example:q in pi} where we saw that $\pi = 35142$ contains 
  $q = 213$ as a pattern with only one occurrence. There is only one occurrence 
  of $q$ in $\sigma = 53241$ as well, namely $324$. Figure~\ref{figure:p not in 
  pi and p in sigma} shows us two things. One the left side, we see that $p = 
  \left( 213, \left\{ \boks{1,2}, \boks{1,3}, \boks{2,2} \right\} \right)$ is 
  \emph{not} contained in $\pi$, because the point $(2,5) \in \Gr{\pi}$ lies 
  inside $R_{13}$. On the right side we see that $\sigma$ contains $p$, since 
  for the only occurrence of $q$ in $\sigma$ the shaded area does not contain 
  any of the points of $\Gr{\sigma}$. 

  \begin{figure}[htbp]
    \center
    \begin{tikzpicture}[scale=\pattdispscale, baseline={([yshift=-3pt]current bounding box.center)}]
      \useasboundingbox (0.0,-0.1) rectangle (5+1.4,5+1.1);
      
      \foreach [count=\x] \y in {3, 5, 1, 4, 2}
          \filldraw (\x,\y) circle (4pt);
      
      \foreach \x/\y in {1/3, 3/1, 4/4}
          \draw[thick,red] (\x,\y) circle (7pt);

      \foreach \xa/\ya/\xb/\yb in {1/4/3/6, 1/3/3/4, 3/3/4/4}
          \draw[line width=1pt] (\xa+0.1,\ya+0.1) rectangle (\xb-0.1,\yb-0.1);

      \foreach \x/\y in {1/3, 1/4, 1/5, 2/3, 2/4, 2/5, 3/3}
          \fill[pattern color = black!65, pattern=north east lines] (\x,\y) rectangle +(1,1);
      
      \draw[very thin] (1,0.01) -- (1,5.99);
      \draw[very thin] (3,0.01) -- (3,5.99);
      \draw[very thin] (4,0.01) -- (4,5.99);
      \draw[very thin] (0.01,1) -- (5.99,1);
      \draw[very thin] (0.01,3) -- (5.99,3);
      \draw[very thin] (0.01,4) -- (5.99,4);

      \draw[densely dotted, line width=0.6pt] (2,0.01) -- (2,5.99);
      \draw[densely dotted, line width=0.6pt] (5,0.01) -- (5,5.99);
      \draw[densely dotted, line width=0.6pt] (0.01,2) -- (5.99,2);
      \draw[densely dotted, line width=0.6pt] (0.01,5) -- (5.99,5);
    \end{tikzpicture}
    \quad\quad
    \begin{tikzpicture}[scale=\pattdispscale, baseline={([yshift=-3pt]current bounding box.center)}]
      \useasboundingbox (0.0,-0.1) rectangle (5+1.4,5+1.1);
      
      \foreach [count=\x] \y in {5, 3, 2, 4, 1}
          \filldraw (\x,\y) circle (4pt);
      
      \foreach \x/\y in {2/3, 3/2, 4/4}
          \draw[thick,red] (\x,\y) circle (7pt);

      \foreach \xa/\ya/\xb/\yb in {2/4/3/6, 2/3/3/4, 3/3/4/4}
          \draw[line width=1pt] (\xa+0.1,\ya+0.1) rectangle (\xb-0.1,\yb-0.1);

      \foreach \x/\y in {2/3, 2/4, 2/5, 3/3}
          \fill[pattern color = black!65, pattern=north east lines] (\x,\y) rectangle +(1,1);
      
      \draw[very thin] (2,0.01) -- (2,5.99);
      \draw[very thin] (3,0.01) -- (3,5.99);
      \draw[very thin] (4,0.01) -- (4,5.99);
      \draw[very thin] (0.01,2) -- (5.99,2);
      \draw[very thin] (0.01,3) -- (5.99,3);
      \draw[very thin] (0.01,4) -- (5.99,4);

      \draw[densely dotted, line width=0.6pt] (1,0.01) -- (1,5.99);
      \draw[densely dotted, line width=0.6pt] (5,0.01) -- (5,5.99);
      \draw[densely dotted, line width=0.6pt] (0.01,1) -- (5.99,1);
      \draw[densely dotted, line width=0.6pt] (0.01,5) -- (5.99,5);
    \end{tikzpicture}
    \caption{The mesh pattern $p$ from Figure~\ref{figure: Gr(213, shading)} is 
    not contained in $\pi$ (left), but is contained in $\sigma$ (right)}
    \label{figure:p not in pi and p in sigma}
  \end{figure}
\end{example}

Our definitions of $\Av{p}$ and $\Co{p}$ naturally extend to be the set of all 
permutations that avoid and contain $p$, respectively. Similarly $\Av{M} = 
\bigcap\limits_{m \in M}\Av{m}$ and $\Co{M} = \S \setminus \Av{M}$ for a set $M$ 
of mesh patterns.

\emph{Vincular} \cite{babson_generalized_2000}, \emph{covincular} 
\cite{bean_enumerations_2017} and \emph{bivincular} 
\cite{bousquet-melou_2+2-free_2010} permutation patterns are special cases of 
the more general mesh patterns where there are restrictions on the occurrence to 
have adjacent indices, values or both in the permutation. We write them as a 
triple $(\pi, X, Y)$ where $\pi$ is the underlying permutation of length $n$ and 
$X,Y \subseteq \llbracket 0, n \rrbracket$ denote which columns and rows are 
shaded in the mesh pattern. Using our notation, this is equal to the mesh 
pattern \[ (\pi, R) \text{ with } R = \bigcup_{ x \in X }{ \left\{ x \right\} 
\times \llbracket 0, n \rrbracket } \cup \bigcup_{ y \in Y }{ \llbracket 0, n 
\rrbracket \times \left\{ y \right\} } .\] If $X \neq \emptyset$ and $Y = 
\emptyset$ it is a vincular pattern, if $X = \emptyset$ and $Y \neq \emptyset$ 
it is a covincular pattern and if neither $X$ or $Y$ is the empty set it is a 
bivincular pattern. See Figure~\ref{figure: Examples of vincular, covincular 
and bivincular permutation patterns} for examples of a vincular, a covincular 
and a bivincular permutation pattern.

\begin{figure}[htbp]
  \center
  \mpattern{scale=\patttablescale}{ 3 }{ 1/1, 2/2, 3/3 }{1/0, 1/1, 1/2, 1/3, 3/0, 3/1, 3/2, 3/3 }
  \mpattern{scale=\patttablescale}{ 3 }{ 1/2, 2/3, 3/1 }{0/2, 1/2, 2/2, 3/2 }
  \mpattern{scale=\patttablescale}{ 3 }{ 1/3, 2/1, 3/2 }{2/0, 2/1, 2/2, 2/3, 0/1, 1/1, 3/1 }
  \caption{$(123, \{ 1, 3 \}, \emptyset)$, $(231, \emptyset, \{ 2 \})$ and $(312, \{ 2 \}, \{ 1 \})$}
  \label{figure: Examples of vincular, covincular and bivincular permutation patterns}
\end{figure}

So far we have defined the concept of permutations containing permutations and 
mesh patterns. \textcite{tannock_equivalence_2018} defined mesh pattern 
containment of mesh patterns, recalled below.

\begin{definition}[\citeauthor{tannock_equivalence_2018}]
  A mesh pattern $q = (\kappa, T)$ \emph{contains} a mesh pattern $p = 
  (\tau, R)$ as a \emph{subpattern} if $\kappa$ contains $p$ and $\left( 
  \bigcup_{(i,j) \in R}{R_{ij}} \right) \subseteq T$.
\end{definition}

If $p$ does not contain $q$ we say that $p$ \emph{avoids} $q$.


%%%%%%%%%%%%%%%%%%%%%%%%%%%%%%%%%%%%%%%%%%%%%%%%%%%%%%%%%%%%%%%%%%%%%%%%%%%%%%%%
\subsection{Mesh tilings}

\begin{definition}\label{definition:block}
  A \emph{block}\footnote{In the Python source code this is called a 
  \emph{cell}.} $\mf{B}$ is a tuple $(\mc{O}, \mc{R})$ of two sets called 
  \emph{obstructions} and \emph{requirements} of mesh patterns that 
  \emph{generates} the set of permutations \[ \Av{\mc{O}} \cap \Co{\mc{R}} .\]
  The block $\overset{\leftharpoonup}{\mf{B}} = (\mc{R}, \mc{O})$ is called the 
  \emph{flipped} block (of $\mf{B}$).
\end{definition}

We use the term ``block'' interchangeably for the tuple and the set of 
permutations that it generates. 

\begin{example}
  Some notable blocks include the \emph{point block} $(\left\{ 12, 21 \right\}, 
  \left\{ 1 \right\})$, the \emph{empty block} $(\left\{ 1 \right\}, \emptyset)$ 
  and the \emph{free block} $(\emptyset, \emptyset)$ that generate the sets 
  $\left\{ 1 \right\}$, $\left\{ \epsilon \right\}$ and $\S$ respectively.
\end{example}

Before moving on to define mesh tilings we review generalized grid classes 
introduced by \textcite{vatter_small_2011}. Given a permutation $\pi$ of length 
$n$ and sets $X, Y \subseteq \llbracket n \rrbracket$, we denote with $\pi( X 
\times Y )$ the standardization of the subsequence of $\pi$ with indices from 
$X$ and values in $Y$. For example $25134([1, 4] \times [2, 5]) = \st(253) = 
132$.

Now suppose $\mc{M}$ is a $c \times r$ matrix (indexed from left to right and 
bottom to top) whose entries are sets of permutations. An 
\emph{$\mc{M}$-gridding} of the permutation $\pi$ of length $n$ is a pair of 
sequences $1 = c_1 \leq \cdots \leq c_{t+1} = n + 1$ (the column divisions) 
and $1 = r_1 \leq \cdots \leq r_{u+1} = n + 1$ (the row divisions) such that 
$\pi(\left[c_k, c_{k+1} \right) \times \left[r_l, r_{l+1} \right))$ is either 
empty or is a member of $\mc{M}_{k,l}$ for all $k \in \llbracket t \rrbracket$ 
and $l \in \llbracket u \rrbracket$. The \emph{generalized grid class} of 
$\mc{M}$, $\Grid{\mc{M}}$, is the set of all permutations with an 
$\mc{M}$-gridding.

\begin{definition}
  A \emph{mesh tiling} $\mc{M}$ is a $c \times r$ matrix of \emph{blocks} (as in 
  Definition~\ref{definition:block}), representing the set of permutations 
  $\Grid{\mc{M}}$.
\end{definition}

A $1 \times 1$ mesh tiling generates the same set of permutations as the (only) 
block in the mesh tiling.

\begin{example}
  The $2 \times 2$ mesh tiling (depicted in Figure~\ref{figure:mesh tiling of 
  perms starting with one}) with empty blocks in coordinates $(0, 1)$ and 
  $(1, 0)$, a point block at $(0, 0)$ and a free block at $(1, 1)$ generates 
  the set of permutations starting with $1$.

  \begin{figure}[htbp]
    \center
    \strule{\pattdispscale}{2}{2}{(0,0)/\point{2pt}, (1,1)/$\S$}
    \caption{A mesh tiling generating all permutations starting with one}
    \label{figure:mesh tiling of perms starting with one}
  \end{figure}
\end{example}


%%%%%%%%%%%%%%%%%%%%%%%%%%%%%%%%%%%%%%%%%%%%%%%%%%%%%%%%%%%%%%%%%%%%%%%%%%%%%%%%
\section{Implementation\label{mesh tilings implementation}}

With mesh tilings we are seeking to find covers for avoidance sets of 
permutations of the form $\Av{M}$ where $M$ is a set of mesh patterns. Often we 
are interested in the complementary set $\Co{M}$ as well because a cover for it 
might help us understand the structure of the permutation class. In practice, 
therefore, we usually apply \CombCov\ on a single $1 \times 1$ mesh tilings 
hoping to find a cover of the form of one or more mesh tilings of various sizes. 
Our implementation expects the root object to be a $1 \times 1$ mesh tiling but 
it works as well for mesh tilings of larger sizes.

The framework requires us to implement two main functions for our mesh tiling 
object in Python. Given a $c \times r$ mesh tiling $\mc{M}$ these methods are:
\begin{itemize}[(i)]
  \item \texttt{get\_elmnts}: Takes in a parameter $l$ and outputs all elements 
    of length $l$ generated $\mc{M}$ in a consistent order. This is easy to 
    describe mathematically but harder to implement programmatically. We refer 
    to the original implementation by \textcite{bean_permstruct_2017} for 
    further details.
  \item \texttt{get\_subrules}: Generates the subrules $S_i$ of the root object 
    $\mc{M}$. Recall that not all subrules need to be \emph{valid}, but from a 
    performance aspect it preferably does not generate too many invalid rules. 
    Algorithm~\ref{algorithm:mesh tiling subrule generation} shows in 
    pseudo-code how we implemented the subrule generation that yielded the 
    results in Section~\ref{mesh tilings results}. This is the product of 
    several iterations where we edited the algorithm to find more covers while 
    keeping the search space small enough for tolerable running times.
\end{itemize}

\begin{algorithm}[ht]
  \caption{Mesh tiling subrule generation}
  \begin{algorithmic}
  \REQUIRE $1 \times 1$ mesh tiling $\mc{M}$ and positive integers \texttt{max\_columns}, \texttt{max\_rows} and \texttt{max\_active\_blocks}
  \STATE $\mc{C} \leftarrow \mc{M}_{1, 1}$
  \STATE $P \leftarrow \emptyset$
  \FORALL{patterns $p$ in the union of the obstructions and requirements of $\mc{C}$}
    \IF{$p$ is a permutation}
      \STATE append all subpermutations of $p$ to $P$
    \ELSIF{$p$ is a mesh pattern}
      \STATE append all subpatterns of $p$ to $P$
    \ENDIF
  \ENDFOR
  \STATE $C \leftarrow$ the set of $\mc{C}$, $\overset{\leftharpoonup}{\mc{C}}$, the point block and the free block
  \FORALL{patterns $p$ in $P$}
    \STATE append the block $\left( \left\{ p \right\}, \emptyset \right)$ to $C$
  \ENDFOR
  \STATE $M \leftarrow$ the set of the $1 \times 1$ mesh tiling with the empty block
  \FORALL{$i$ such that $1 \leq i \leq \texttt{max\_columns}$}
  \FORALL{$j$ such that $1 \leq j \leq \texttt{max\_rows}$}
  \FORALL{$a$ such that $i * j \leq a \leq \texttt{max\_active\_blocks}$}
    \FORALL{combinations of $a$ blocks (with repetition) from $C$}
      \FORALL{combinations of $a$ squares in an $i \times j$ grid of which there is no column or row with no selected square}
        \STATE append the $i \times j$ mesh tiling with the $a$ blocks placed in the chosen squares to $M$
      \ENDFOR
    \ENDFOR
  \ENDFOR
  \ENDFOR
  \ENDFOR
  \RETURN $M$
  \end{algorithmic}
  \label{algorithm:mesh tiling subrule generation}
\end{algorithm}



%%%%%%%%%%%%%%%%%%%%%%%%%%%%%%%%%%%%%%%%%%%%%%%%%%%%%%%%%%%%%%%%%%%%%%%%%%%%%%%%
\section{Results\label{mesh tilings results}}

Our aim here is twofold. One is to reproduce old results by finding covers for 
sets of permutations avoiding a set of mesh patterns (we will call this 
\emph{avoidance sets} as they are not necessarily permutation classes) and 
compare them to the published results. The other is to try to find covers for 
avoidance sets that are not already known and thus contribute new knowledge to 
the field of enumerative combinatorics.

Just as in Section~\ref{words results} all results presented here were obtained 
using version \texttt{v0.6.3} of \CombCov\ and \texttt{max\_elmnt\_size = 7}.
As these problems demanded more computing power we ran them on the computer 
cluster \emph{Garpur}\footnote{See \url{http://ihpc.is/garpur/} for details.} 
managed by the University of Iceland. The execution times were on the scale of 
hours and days with a hard 2-week time limit after which the execution was 
stopped and deemed unsuccessful.

We inputted 620 different avoidance sets into \CombCov\ along with their 
complementary containment sets for a total of 1240 unique jobs with the 
framework. In each job we tried multiple and increasing values for 
\texttt{max\_columns}, \texttt{max\_rows} and \texttt{max\_active\_blocks}, 
encoded as a triple $(c, r, a)$. Their initial values were set to $(2, 2, 3)$, 
and if a cover was not found we next tried $(3, 3, 3$), after that $(4, 4, 4)$, 
then $(4, 4, 5)$ and lastly $(5, 5, 5)$. We won't summarize all of the results 
here but instead highlight a few interesting covers and discuss their 
significance.


%%%%%%%%%%%%%%%%%%%%%%%%%%%%%%%%%%%%%%%%%%%%%%%%%%%%%%%%%%%%%%%%%%%%%%%%%%%%%%%%
\subsection{Vincular and covincular length 3 mesh patterns\label{Vincular and covincular length 3 mesh patterns results}}

\textcite{claesson_generalized_2001} studied \emph{generalized permutation 
patterns} (which later became called vincular patterns) of length 3 and 
enumerated the permutation sets avoiding one or two such patterns. He used a 
notation of letters and dashes to indicate shaded columns, e.g.,
$a\text{-}bc = \mpattern{scale=\patttextscale}{ 3 }{ 1/1, 2/2, 3/3 }{ 2/0, 2/1, 2/2, 2/3 }$ and
$ca\text{-}b = \mpattern{scale=\patttextscale}{ 3 }{ 1/3, 2/1, 3/2 }{ 1/0, 1/1, 1/2, 1/3 }$.

In Figure~\ref{figure:Cleasson Av(a-bc)} we show the cover which \CombCov\ finds 
for an avoidance set that is enumerated by the \emph{Bell numbers} (OEIS 
sequence number \oeis{A000110}), meaning that $\left| \Avn{\mc{A}} \right|$ is 
the $n$-th Bell number, $B_n$. 

\begin{figure}[htbp]
  \center
    \begin{tabular}{ r c l l }
    $\mc{A} = \Av{ \mpattern{scale=\patttextscale}{ 3 }{ 1/1, 2/2, 3/3 }{ 2/0, 2/1, 2/2, 2/3 } }$ & $=$ & $ 
    \strule{\pattdispscale}{1}{1}{} \mediumsqcup
    \strule{\pattdispscale}{3}{2}{
      (0,1)/$\mc{A}$,
      (1,0)/\point{2pt}, 
      (2,1)/$\mc{B}$
    }$ & $\mc{B} = \Av{ \mpattern{scale=\patttextscale}{ 2 }{ 1/1, 2/2 }{} }$ 
  \end{tabular}
  \caption{\CombCov\ cover of $\Av{a\text{-}bc}$}
  \label{figure:Cleasson Av(a-bc)}
\end{figure}

It's easy to see that the cover indeed generates the permutations that avoid 
$\mpattern{scale=\patttextscale}{ 3 }{ 1/1, 2/2, 3/3 }{ 2/0, 2/1, 2/2, 2/3 }$. 
More interesting is deriving the \emph{exponential generating function (EGF)} 
$F = \sum_{n \geq 0} {a_n \over n!} x^n$ and show that $a_n = B_n$, proving that 
the sequence is enumerated by the Bell numbers.

We start by noting that $\left| \Avn{12} \right| = 1$ for all $n$ so $\mc{B}$ 
in Figure~\ref{figure:Cleasson Av(a-bc)} is enumerated by $(1)_n$ and has EGF 
$\sum_{n \geq 0} {1 \over n!} x^n = e^x$.
This gives us

\begin{align*}
  \text{}     \qquad & F = 1 + \int F e^x dx \\
  \text{i.e.} \qquad & \frac{d}{dx} F = F e^x \\
  \text{i.e.} \qquad & \int \frac{1}{F} dF = \int e^x dx \\
  \text{i.e.} \qquad & \ln(F) = e^x + C \\
  \text{i.e.} \qquad & F = A e^x
\end{align*}

We have shown that $\sum_{n \geq 0} {a_n \over n!} x^n = Ae^x$ for some constant 
$A$. We know that there is only one permutation of length zero in $\mc{A}$ so 
putting $a_0 = 1$ into the equation at $x = 0$ gives us $1 = Ae$ i.e.\@ 
$A = e^{-1}$ so $F = e^{e^x - 1}$ which is indeed the EGF for the Bell numbers.

In Figure~\ref{figure:Cleasson Av(a-bc, a-cb)} we present the cover \CombCov\ 
found for an avoidance set that is enumerated with $I_n$ (\oeis{A000085}), the 
number of \emph{involutions} in $\S_n$.

\begin{figure}[htbp]
  \center
    \begin{tabular}{ r c l l }
    $\mc{A} = \Av{ \mpattern{scale=\patttextscale}{ 3 }{ 1/1, 2/2, 3/3 }{ 2/0, 2/1, 2/2, 2/3 }, \mpattern{scale=\patttextscale}{ 3 }{ 1/1, 2/3, 3/2 }{ 2/0, 2/1, 2/2, 2/3 } }$ & $=$ & $ 
    \strule{\pattdispscale}{1}{1}{} \mediumsqcup
    \strule{\pattdispscale}{2}{2}{
      (0,1)/$\mc{A}$, 
      (1,0)/\point{2pt}
    } \mediumsqcup
    \strule{\pattdispscale}{3}{2}{
      (0,1)/$\mc{A}$, 
      (1,0)/\point{2pt},
      (2,1)/\point{2pt}
    }$ & 
  \end{tabular}
  \caption{\CombCov\ cover of $\Av{a\text{-}bc, a\text{-}cb}$}
  \label{figure:Cleasson Av(a-bc, a-cb)}
\end{figure}

Similarly we can easily see that the cover is indeed correct and derive the 
exponential generating function to prove the enumeration.

\textcite{bean_enumerations_2017} studied the set of permutations simultaneously 
avoiding a vincular and covincular mesh pattern of length 3. An interesting 
result is that \CombCov\ finds the same kind of cover for both $\Av{ 
\mpattern{scale=\patttextscale}{ 3 }{ 1/1, 2/2, 3/3 }{ 1/0, 1/1, 1/2, 1/3 }, 
\mpattern{scale=\patttextscale}{ 3 }{ 1/2, 2/1, 3/3 }{ 0/2, 1/2, 2/2, 3/2 } 
}$ and $\Av{ \mpattern{scale=\patttextscale}{ 3 }{ 1/1, 2/2, 3/3 }{
0/2, 1/2, 2/2, 3/2}, \mpattern{scale=\patttextscale}{ 3 }{ 1/1, 2/3, 3/2 }{} 
}$ which is shown in Figure~\ref{figure:Covincular Motzkin numbers}. These 
avoidance sets are enumerated by the \emph{Motzkin} numbers $M_n$ 
(\oeis{A001006}) and it is easy to see from the cover that the generating 
function fulfills $F(x) = 1 + xF(x) + x^2F(x)^2$ which indeed gives us the 
coefficients $M_n$ to $x^n$.

\begin{figure}[htbp]
  \center
    \begin{tabular}{ r c l l }
    $\mc{A} = \Av{ \mpattern{scale=\patttextscale}{ 3 }{ 1/1, 2/2, 3/3 }{ 1/0, 1/1, 1/2, 1/3 }, \mpattern{scale=\patttextscale}{ 3 }{ 1/2, 2/1, 3/3 }{ 0/2, 1/2, 2/2, 3/2 } }$ & $=$ & $
    \strule{\pattdispscale}{1}{1}{} \mediumsqcup
    \strule{\pattdispscale}{2}{2}{
      (0,1)/$\mc{A}$, 
      (1,0)/\point{2pt}
    } \mediumsqcup
    \strule{\pattdispscale}{4}{4}{
      (0,3)/$\mc{A}$, 
      (1,0)/\point{2pt},
      (2,2)/\point{2pt},
      (3,1)/$\mc{A}$
    } $ & \\
    $\mc{B} = \Av{ \mpattern{scale=\patttextscale}{ 3 }{ 1/1, 2/2, 3/3 }{0/2, 1/2, 2/2, 3/2}, \mpattern{scale=\patttextscale}{ 3 }{ 1/1, 2/3, 3/2 }{} }$ & $=$ & $
    \strule{\pattdispscale}{1}{1}{} \mediumsqcup
    \strule{\pattdispscale}{2}{2}{
      (0,1)/$\mc{B}$, 
      (1,0)/\point{2pt}
    } \mediumsqcup
    \strule{\pattdispscale}{4}{4}{
      (0,3)/$\mc{B}$, 
      (1,0)/\point{2pt},
      (2,2)/\point{2pt},
      (3,1)/$\mc{B}$
    } $ &
  \end{tabular}
  \caption{\CombCov\ cover of avoidance sets that gives rise to the Motzkin numbers}
  \label{figure:Covincular Motzkin numbers}
\end{figure}

Other successful covers in this paper include 
$\Av{ \mpattern{scale=\patttextscale}{ 3 }{ 1/1, 2/3, 3/2 }{ 1/0, 1/1, 1/2, 1/3 }, \mpattern{scale=\patttextscale}{ 3 }{ 1/2, 2/3, 3/1 }{ 0/2, 1/2, 2/2, 3/2 } }$ 
(Figure~\ref{figure:Covincular Av(132_1_e, 231_e_2)}, \oeis{A011782}) and 
$\Av{ \mpattern{scale=\patttextscale}{ 3 }{ 1/1, 2/2, 3/3 }{ 2/0, 2/1, 2/2, 2/3 }, \mpattern{scale=\patttextscale}{ 3 }{ 1/2, 2/3, 3/1 }{ 0/1, 1/1, 2/1, 3/1 } }$ 
(Figure~\ref{figure:Covincular Av(123_2_e, 231_e_1)}, \oeis{A152947}).

\begin{figure}[htbp]
  \center
  \begin{tabular}{ r c l l }
    $\Av{ \mpattern{scale=\patttextscale}{ 3 }{ 1/1, 2/3, 3/2 }{ 1/0, 1/1, 1/2, 1/3 }, \mpattern{scale=\patttextscale}{ 3 }{ 1/2, 2/3, 3/1 }{ 0/2, 1/2, 2/2, 3/2 } }$ & $=$ & $
    \strule{\pattdispscale}{1}{1}{} \mediumsqcup
    \strule{\pattdispscale}{3}{2}{
      (0,1)/$\mc{B}$, 
      (1,0)/\point{2pt},
      (2,1)/$\mc{C}$
    } $ & \begin{tabular}{@{}c@{}}
            $\mc{B} = \Av{ \mpattern{scale=\patttextscale}{ 2 }{ 1/1, 2/2 }{} }$ \\
            $\mc{C} = \Av{ \mpattern{scale=\patttextscale}{ 2 }{ 1/2, 2/1 }{} }$
          \end{tabular}
  \end{tabular}
  \caption{\CombCov\ cover for a pair of vincular and covincular mesh patterns}
  \label{figure:Covincular Av(132_1_e, 231_e_2)}
\end{figure}

\begin{figure}[htbp]
  \center
  \begin{tabular}{ r c l l }
    $\mc{A} = \Av{ \mpattern{scale=\patttextscale}{ 3 }{ 1/1, 2/2, 3/3 }{ 2/0, 2/1, 2/2, 2/3 }, \mpattern{scale=\patttextscale}{ 3 }{ 1/2, 2/3, 3/1 }{ 0/1, 1/1, 2/1, 3/1 } }$ & $=$ & $
    \strule{\pattdispscale}{1}{1}{} \mediumsqcup
    \strule{\pattdispscale}{2}{2}{
      (0,1)/\point{2pt}, 
      (1,0)/$\mc{A}$
    } \mediumsqcup
    \strule{\pattdispscale}{4}{4}{
      (0,1)/$\mc{B}$, 
      (1,0)/\point{2pt},
      (2,3)/$\mc{B}$,
      (3,2)/\point{2pt}
    }$ & $\mc{B} = \Av{ \mpattern{scale=\patttextscale}{ 2 }{ 1/1, 2/2 }{} }$
  \end{tabular}
  \caption{\CombCov\ cover for another pair of vincular and covincular mesh patterns}
  \label{figure:Covincular Av(123_2_e, 231_e_1)}
\end{figure}

We note that $\mc{B} = \left\{ \epsilon, 1, 21, 321, \ldots \right\}$ is the set 
of all decreasing permutations so the generating function for $\Av{ 
\mpattern{scale=\patttextscale}{ 3 }{ 1/1, 2/2, 3/3 }{ 2/0, 2/1, 2/2, 2/3 }, 
\mpattern{scale=\patttextscale}{ 3 }{ 1/2, 2/3, 3/1 }{ 0/1, 1/1, 2/1, 3/1 } 
}$ in Figure~\ref{figure:Covincular Av(123_2_e, 231_e_1)} fulfills \[ F(x) 
= 1 + xF(x) + \frac{x^2}{(1 - x)^2} \] which we can solve to find \[ F(x) = 
\frac{2x^2 - 2x + 1}{(1 - x)^3} \] which gives us the coefficients $1 + {n 
\choose 2}$ to $x^n$.


%%%%%%%%%%%%%%%%%%%%%%%%%%%%%%%%%%%%%%%%%%%%%%%%%%%%%%%%%%%%%%%%%%%%%%%%%%%%%%%%
\subsection{Length 2 mesh patterns\label{Length 2 mesh patterns results}}

\textcite{hilmarsson_wilf-classification_2015} classified all mesh patterns of 
length 2 into \emph{Wilf-equivalence} classes, i.e., the classes of the same 
enumerations. Some notable covers are shown in 
Figures~\ref{figure:WilfShort-1234678}, \ref{figure:WilfShort-234678} and 
\ref{figure:WilfShort-containment}.

\begin{figure}[htbp]
  \center
  \begin{tabular}{ r c l l }
    $\Av{ \mpattern{scale=\patttextscale}{ 2 }{ 1/2, 2/1 }{ 0/0, 0/1, 0/2, 1/0, 1/2, 2/0, 2/1 } }$ & $=$ & $
    \strule{\pattdispscale}{1}{1}{} \mediumsqcup
    \strule{\pattdispscale}{2}{2}{
      (0,0)/\point{2pt}, 
      (1,1)/$\mc{B}$
    } \mediumsqcup
    \strule{\pattdispscale}{2}{2}{
      (0,1)/\point{2pt}, 
      (1,0)/\point{2pt},
      (1,1)/$\S$
    }$ & $\mc{B} = \Av{ \mpattern{scale=\patttextscale}{ 1 }{ 1/1 }{ 0/1, 1/0 } }$ \\
    $\Co{ \mpattern{scale=\patttextscale}{ 2 }{ 1/2, 2/1 }{ 0/0, 0/1, 0/2, 1/0, 1/2, 2/0, 2/1 } }$ & $=$ & $
    \strule{\pattdispscale}{4}{4}{
      (0,0)/\point{2pt}, 
      (1,1)/$\S$,
      (2,2)/\point{2pt},
      (3,3)/$\mc{B}$
    }$ & 
    \end{tabular}
  \caption{Avoidance and containment covers for a length 2 mesh pattern}
  \label{figure:WilfShort-1234678}
\end{figure}

\begin{figure}[htbp]
  \center
  \begin{tabular}{ r c l l }
    $\Av{ \mpattern{scale=\patttextscale}{ 2 }{ 1/2, 2/1 }{ 0/1, 0/2, 1/0, 1/2, 2/0, 2/1 } }$ & $=$ & $
    \strule{\pattdispscale}{1}{1}{
      (0,0)/$\mc{B}$
    } \mediumsqcup
    \strule{\pattdispscale}{3}{3}{
      (0,0)/$\mc{B}$, 
      (1,1)/\point{2pt},
      (2,2)/$\mc{B}$
    }$ & $\mc{B} = \Av{ \mpattern{scale=\patttextscale}{ 1 }{ 1/1 }{ 0/1, 1/0 } }$ \\
    $\Co{ \mpattern{scale=\patttextscale}{ 2 }{ 1/2, 2/1 }{ 0/1, 0/2, 1/0, 1/2, 2/0, 2/1 } }$ & $=$ & $
    \strule{\pattdispscale}{5}{5}{
      (0,0)/$\S$,
      (1,1)/\point{2pt}, 
      (2,2)/$\mc{B}$,
      (3,3)/\point{2pt},
      (4,4)/$\mc{B}$
    }$
  \end{tabular}
  \caption{Avoidance and containment covers for another length 2 mesh pattern}
  \label{figure:WilfShort-234678}
\end{figure}

\begin{figure}[htbp]
  \center
  \begin{tabular}{ r c l l }
    $\Co{ \mpattern{scale=\patttextscale}{ 2 }{ 1/2, 2/1 }{ 0/0, 0/1, 1/0, 1/2, 2/1, 2/2 } }$ & $=$ & $
    \strule{1}{5}{5}{
      (0,4)/$\mc{C}$, 
      (1,1)/\point{2pt},
      (2,2)/$\S$,
      (3,3)/\point{2pt},
      (4,0)/$\S$
    }$ & $\mc{C} = \Av{ \mpattern{scale=\patttextscale}{ 2 }{ 1/2, 2/1 }{ 0/0, 0/1, 1/0, 1/2, 2/1, 2/2 } }$ \\
    $\Co{ \mpattern{scale=\patttextscale}{ 2 }{ 1/2, 2/1 }{ 0/0, 0/1, 1/1, 1/2, 2/0, 2/1 } }$ & $=$ & $
    \strule{1}{5}{4}{
      (0,3)/$\S$, 
      (1,1)/\point{2pt},
      (2,0)/$\mc{D}$,
      (3,2)/\point{2pt},
      (4,3)/$\S$
    }$ & $\mc{D} = \Av{ \mpattern{scale=\patttextscale}{ 2 }{ 1/2, 2/1 }{ 0/0, 0/1, 1/1, 1/2, 2/0, 2/1 } }$ \\
    $\Co{ \mpattern{scale=\patttextscale}{ 2 }{ 1/2, 2/1 }{ 0/0, 0/1, 1/1, 1/2, 2/0, 2/2 } }$ & $=$ & $
    \strule{1}{5}{5}{
      (0,4)/$\S$, 
      (1,1)/\point{2pt},
      (2,0)/$\S$,
      (3,3)/\point{2pt},
      (4,2)/$\S$
    }$ &
  \end{tabular}
  \caption{Containment covers for some length 2 mesh pattern}
  \label{figure:WilfShort-containment}
\end{figure}


%%%%%%%%%%%%%%%%%%%%%%%%%%%%%%%%%%%%%%%%%%%%%%%%%%%%%%%%%%%%%%%%%%%%%%%%%%%%%%%%
\subsection{Bivincular permutation patterns\label{Bivincular permutation patterns results}}

We now focus on some results of bivincular permutation patterns replicated from 
\textcite{parviainen_wilf_2009}. Among them is the cover
\[
  \Av{ \mpattern{scale=\patttextscale}{ 3 }{ 1/1, 2/3, 3/2 }{ 0/3, 1/3, 2/3, 3/3 } } =
  \strule{1}{1}{1}{} \mediumsqcup
  \strule{1}{3}{3}{
    (0,1)/$\S$,
    (1,2)/\point{2pt},
    (2,0)/$\S$
  }
\]
(enumeration \oeis{A003149}) which is especially interesting when comparing it 
to the very well known
\[
  \Av{ \mpattern{scale=\patttextscale}{ 3 }{ 1/1, 2/3, 3/2 }{} } = 
  \strule{1}{1}{1}{} \mediumsqcup
  \strule{1}{3}{3}{
    (0,1)/$\mc{A}$,
    (1,2)/\point{2pt},
    (2,0)/$\mc{A}$
  }
\]
(enumerated by the \emph{Catalan numbers}, \oeis{A000108}) with $\mc{A} = \Av{ 
\mpattern{scale=\patttextscale}{ 3 }{ 1/1, 2/3, 3/2 }{} }$. The 
difference is that $\Av{ \mpattern{scale=\patttextscale}{ 3 }{ 1/1, 2/3, 
3/2 }{ 0/3, 1/3, 2/3, 3/3 } }$ is the set of permutations $\pi$ avoiding 
$231$ in which the $3$ in the occurrence of $231$ is the \emph{topmost point} of 
$\pi$. This is already secured by the point block in the second mesh tiling in 
the cover of $\Av{ \mpattern{scale=\patttextscale}{ 3 }{ 1/1, 2/3, 3/2 }{ 
0/3, 1/3, 2/3, 3/3 } }$ and explains why it includes $\S$ blocks instead 
of the root.

In addition to finding covers for the results presented in the paper, we 
searched for covers for all possible length 3 bivincular mesh patterns, 
discarding symmetries. That made for 163 more avoidance sets to find covers for, 
out of which 79 were successful. Some interesting covers are shown in 
Figure~\ref{figure:WilfBiVincular-avoidance}.

\begin{figure}[htbp]
  \center
  \begin{tabular}{ r c l l }
    $\Av{ \mpattern{scale=\patttextscale}{ 3 }{ 1/1, 2/2, 3/3 }{ 0/0, 0/1, 0/2, 0/3, 1/3, 2/3, 3/3 } }$ & $=$ & $
    \strule{\pattdispscale}{1}{1}{} \mediumsqcup
    \strule{\pattdispscale}{2}{3}{
      (0,1)/\point{2pt},
      (1,0)/$\S$,
      (1,2)/$\mc{B}$
    }$ & $\mc{B} = \Av{ \mpattern{scale=\patttextscale}{ 2 }{ 1/1, 2/2 }{ 0/0, 0/1, 0/2 } }$ \\
    $\Av{ \mpattern{scale=\patttextscale}{ 3 }{ 1/1, 2/2, 3/3 }{ 0/0, 0/1, 0/2, 0/3, 1/0, 1/1, 1/2, 1/3, 2/0, 2/2, 3/0, 3/2 } }$ & $=$ & $ 
    \strule{\pattdispscale}{1}{1}{
      (0,0)/$\mc{C}$
    } \mediumsqcup
    \strule{\pattdispscale}{2}{2}{
      (0,0)/\point{2pt},
      (1,1)/$\mc{D}$
    }$ & \begin{tabular}{@{}c@{}}
            $\mc{C} = \Av{ \mpattern{scale=\patttextscale}{ 1 }{ 1/1 }{ 0/0, 0/1, 1/0 } }$ \\
            $\mc{D} = \Av{ \mpattern{scale=\patttextscale}{ 2 }{ 1/1, 2/2 }{ 0/1, 0/2, 1/1, 2/1 } }$
          \end{tabular} \\
    $\Av{ \mpattern{scale=\patttextscale}{ 3 }{ 1/1, 2/3, 3/2 }{ 0/0, 0/1, 0/2, 0/3, 1/0, 1/1, 1/2, 1/3, 2/0, 2/1, 2/2, 2/3, 3/0, 3/1, 3/3 } }$ & $=$ & $ 
    \strule{\pattdispscale}{1}{1}{
      (0,0)/$\mc{E}$
    } \mediumsqcup
    \strule{\pattdispscale}{3}{3}{
      (0,0)/\point{2pt},
      (1,2)/\point{2pt},
      (2,1)/$\mc{C}$
    }$ & $\mc{E} = \Av{ \mpattern{scale=\patttextscale}{ 2 }{ 1/1, 2/2 }{ 0/0, 0/1, 0/2, 1/0, 1/1, 1/2, 2/0, 2/2 } }$
  \end{tabular}
  \caption{Avoidance covers for some bivincular mesh patterns of length 3}
  \label{figure:WilfBiVincular-avoidance}
\end{figure}


%%%%%%%%%%%%%%%%%%%%%%%%%%%%%%%%%%%%%%%%%%%%%%%%%%%%%%%%%%%%%%%%%%%%%%%%%%%%%%%%
\subsection{Pattern-avoiding Fishburn permutations\label{Pattern-avoiding Fishburn permutations results}}

\textcite{gil_pattern-avoiding_2018} defined \emph{Fishburn permutations} as 
those the avoid the mesh pattern $p = \mpattern{scale=\patttextscale}{ 3 }{ 
1/2, 2/3, 3/1 }{ 0/1, 1/0, 1/1, 2/1, 1/2, 3/1, 1/3 }$ because $\left| \Avn{p} 
\right| = \xi(n)$ are the \emph{Fishburn numbers} (\oeis{A022493}).

They studied the set of Fishburn permutations avoiding a permutation $\pi$ of 
length 3 or 4 (i.e., the set of permutations simultaneously avoiding $p$ and 
$\pi$). For each of the six permutations $\pi$ of length 3, they enumerated all 
sets $\Av{p, \pi}$ and \CombCov\ finds a cover for five of them. Then the 
authors enumerated some of the sets $\Av{p, \pi}$ for $\pi \in \S_4$ but focused 
mostly on proving that $\left| \Avn{p, \pi} \right| = C_n$ (the \emph{Catalan 
numbers}, \oeis{A000108}) for $\pi \in \left\{ 1234, 1243, 1324, 1423, 2134, 
2143, 3124, 3142 \right\}$. We found a cover only for the last $\pi$, indicating 
that \[ \Av{ \mpattern{scale=\patttextscale}{ 3 }{ 1/2, 2/3, 3/1 }{ 0/1, 
1/0, 1/1, 2/1, 1/2, 3/1, 1/3 }, \mpattern{scale=\patttextscale}{ 4 }{ 1/3, 2/1, 
3/4, 4/2 }{} } = \Av{ \mpattern{scale=\patttextscale}{ 3 }{ 1/2, 
2/3, 3/1 }{} } .\] The only other cover we found with a length 4 
permutation $\pi$ is shown in Figure~\ref{figure:Fishburn-1342}.

\begin{figure}[htbp]
  \center
  \begin{tabular}{ r c l l }
    $\Av{ \mpattern{scale=\patttextscale}{ 3 }{ 1/2, 2/3, 3/1 }{ 0/1, 1/0, 1/1, 2/1, 1/2, 3/1, 1/3 } , 
          \mpattern{scale=\patttextscale}{ 4 }{ 1/1, 2/3, 3/4, 4/2 }{} }$ & $=$ & $
    \strule{\pattdispscale}{1}{1}{} \mediumsqcup
    \strule{\pattdispscale}{3}{2}{
      (0,1)/$\mc{B}$,
      (1,0)/\point{2pt},
      (2,1)/$\mc{C}$
    }$ & \begin{tabular}{@{}c@{}}
            $\mc{B} = \Av{ \mpattern{scale=\patttextscale}{ 2 }{ 1/1, 2/2 }{} }$ \\
            $\mc{C} = \Av{ \mpattern{scale=\patttextscale}{ 3 }{ 1/2, 2/3, 3/1 }{} }$
          \end{tabular}
  \end{tabular}
  \caption{Fishburn permutations avoiding $1342$}
  \label{figure:Fishburn-1342}
\end{figure}

In addition to what the authors did we searched for covers of Fishburn 
permutations avoiding two length 4 permutations $\pi_1$ and $\pi_2$. Out of the 
276 avoidance sets $\Av{p, \pi_1, \pi_2}$ we found covers for 10 of them but 
none is interesting enough to show here.


%%%%%%%%%%%%%%%%%%%%%%%%%%%%%%%%%%%%%%%%%%%%%%%%%%%%%%%%%%%%%%%%%%%%%%%%%%%%%%%%
\subsection{Summary}

Out of the 159 results from the papers that we tried to replicate with \CombCov\ 
it found covers for 49 of them. This is summarized in Table~\ref{table:Summary 
of mesh tiling results}, broken down by paper.

\begin{table}[ht]
    \centering
    \begin{tabular}{ l | c | c | c }
        Paper & Section & Results in paper & Covers found \\ \hline
        \cite{claesson_generalized_2001} & \ref{Vincular and covincular length 3 mesh patterns results} & 6 & 5 \\
        \cite{bean_enumerations_2017} & \ref{Vincular and covincular length 3 mesh patterns results} & 40 & 11 \\
        \cite{hilmarsson_wilf-classification_2015} & \ref{Length 2 mesh patterns results} & 65 & 15 \\
        \cite{parviainen_wilf_2009} & \ref{Bivincular permutation patterns results} & 24 & 11 \\
        \cite{gil_pattern-avoiding_2018} & \ref{Pattern-avoiding Fishburn permutations results} & 24 & 7 
    \end{tabular}
    \caption{Summary of replicating mesh pattern covers}
    \label{table:Summary of mesh tiling results}
\end{table}
