\chapter{Discussion\label{discussion chapter}}

Just as \Struct\ is a useful tool for finding covers of permutation classes we 
have shown that the generalized version \CombCov\ can find covers for sets of 
other types of combinatorial objects. We defined \emph{words} and \emph{mesh 
patterns} and applied \CombCov\ on both known results and new problems with some 
success.


%%%%%%%%%%%%%%%%%%%%%%%%%%%%%%%%%%%%%%%%%%%%%%%%%%%%%%%%%%%%%%%%%%%%%%%%%%%%%%%%
\section{Conclusion\label{sec:conclusions}}

The framework did not find covers for all of the problems that we looked at. 
That is not the framework's fault but ours, as the user writing the algorithm to 
generate subrules. We suspect that the algorithm is theoretically capable of 
finding more covers than was presented in this paper but because of slow running 
times and concrete time limits we did not find more covers. \CombCov\ has proved 
its helpfulness by being able to come up with ideas that previously required 
substantial effort by a human researcher. 


%%%%%%%%%%%%%%%%%%%%%%%%%%%%%%%%%%%%%%%%%%%%%%%%%%%%%%%%%%%%%%%%%%%%%%%%%%%%%%%%
\section{Future work\label{sec:future work}}

Future work includes refining the library with ease-of-use in mind, writing 
documentation and a ``Help get started'' guide. It is our hope that other 
researchers pick it up and apply it to their combinatorial objects of interest 
such as polyominoes, set partitions, alternating sign matrices, ascent sequences 
or something completely different.
