%% ---------------------------------------------------------------
%% $URL: https://repository.cs.ru.is/svn/thesis-template/trunk/ruthesis/latex/DEGREE-NAME-YEAR.tex $
%% $Id: DEGREE-NAME-YEAR.tex 360 2019-02-13 22:04:35Z foley $
%% This is a template LaTeX file for dissertations, theses, or reports at Reykjavík University
%% 
%% Comments and questions can be sent to the RU LaTeX group (latex AT list.ru.is) 
%% ---------------------------------------------------------------

%% METHOD:
%% 0) Read ruthesis/thesis-instructions.pdf
%%    If it is missing, goto https://repository.cs.ru.is/svn/thesis-template/trunk/ruthesis/thesis-instructions.pdf
%% 0.2) Subscribe to the announcements email list at
%%    https://list.ru.is/mailman/listinfo/latex-announcements
%% 1 LaTeX instructions.tex or goto http://afs.rnd.ru.is/project/thesis-template/trunk/ruthesis/latex/instructions.pdf
%% 2) Copy the template files (or unzip) to your working area
%% 3) Rename this file (if needed) with your information e.g. MSC-FOLEY-2007.tex
%% 4) Modify this file to fit your needs (please follow all comments below in the text)
%% 5) For making bibliographies, run "biber".  You can also change
%%    this back to "bibtex".  See below in "Bibliography options".

%%%%%%% CHOOSE ONE OF THESE %%%%%%%%%%%%%%%
%% projectreport: Project report (CS)
%% bachelors: Bachelor of Science thesis
%% masters: Master of Science thesis
%% doctorate: Doctor of Philosophy dissertation
%
%%%%%%% CHOOSE ONE OF THESE %%%%%%
%% 
%% draft: speed up processing by skipping graphics and adding useful
%%     information for editing.  Also sets spacing to double so that it is easier to
%%     write editing marks on paper copy.
%% proof:  proofreading version (final formatting with warnings)
%% final: generate document for submission, removing FIXMEs, and
%%     other markup.  Throw error if any fatal FIXMEs still in document.
%%
%%%%%%% CHOOSE ONE OF THESE IF APPLICABLE %%%%%%
%%
%% deptsse: School of Science and Engineering
%% deptscs: School of Computer Science
%%
%%%%%%%% CHOOSE ANY COMBINATION OF THESE %%%%%%%%%%%%
%%
%% forcegraphics: force graphics, etc. to be included, even in draft mode
%% debug:  writes more messages to the log file, adds debugging output 
%%     and sizing boxes
%% icelandic: thesis is in Icelandic
%% oldstyle:  use the PhD headers and footers from the old CS template
%% online: for online versions (skip blank pages)
\documentclass[online,masters,deptscs,forcegraphics]{ruthesis}

%%%%%%%%%%%%%%%%%%%% TeXStudio Magic Comments %%%%%%%%%%%%%%%%%%%%%
%% These comments that start with "!TeX" modify the way TeXStudio works
%% For details see http://texstudio.sourceforge.neit/manual/current/usermanual_en.html   Section 4.10
%%
%% What encoding is the file in?
% !TeX encoding = UTF-8
%% What language should it be spellchecked?
% !TeX spellcheck = en_US
%% What program should I compile this document with?
% !TeX program = xelatex

%%%%%%%%%%%%%%%%%%%% Bibliography options %%%%%%%%%%%%%%%%%%%%%
%% We suggest switching from bibtex to biblatex/biber because it is better able
%% to deal with Icelandic characters and other bibliography issues
%% As long as you use biblatex instead of bibtex by itself, it will at least
%%  generate a document without errors.
%% !!!If you are using TeXStudio, don't forget to update the bibliography setting!!!
\usepackage[backend=biber,bibencoding=utf8,style=ieee]{biblatex}
%\DeclareLanguageMapping{american}{american-apa}  
% need to declare mapping for style=apa to alphabetize properly
% If you set backend=bibtex, it will use bibtex for processing (old way)
%    this can work with Icelandic characters, but you may get weird results.
%    bibtex does not know how to sort Þ and ð
% if you set backend=biber, you can use UTF8 characters such as Þ and
%     ð  but you will have to remember to switch from using bibtex to 
%     biber in your client
% If you use JabRef, make sure the file is encoded in UTF-8 which is
%    not the default.

%% This tells TeXStudio to use biber
% !TeX TXS-program:bibliography = txs:///biber
%% This also sets the bibliography program for TeXShop and TeXWorks
% !BIB program = biber

% Where is your reference library?
\addbibresource{references.bib}

%%%%%%%%%%%%%%%%%%% CUSTOMIZATIONS %%%%%%%%%%%%%%%%%%%%%%%%%%%%%
%% It is not recommended that you customize this file nor
%% ruthesis.cls.  Just fill in the necessary fields.  You should put
%% your macros and packages into a separate file so that it is easier
%% to use updates to the template.  The custom.sty file was created
%% for this reason.  We load this much later so that it can overrite
%% any existing settings
%\IfFileExists{custom.sty}{\usepackage{custom}}{}


%%%%%%%%%%%%%%%%%%% BJARNI ADDING STUFF %%%%%%%%%%%%%%%%%%%%%%%%%%%%%
\usepackage{graphicx}
\usepackage{tikz}
\usepackage{algorithm,algorithmic}
\usepackage{xargs}
\usepackage{amsmath,amsthm}
\usepackage{mathtools,mathabx}
\usepackage{stmaryrd}
\usepackage{bm}
\usepackage{url}

\newcommand{\mb}[1]{{\mathbb{#1}}}
\newcommand{\mc}[1]{{\mathcal{#1}}}
\newcommand{\mf}[1]{{\mathfrak{#1}}}
\newcommand{\mr}[1]{{\mathrm{#1}}}

\renewcommand{\S}{\mathfrak{S}}
\newcommand{\Av}[1]{{\mathrm{Av}\!\left(#1\right)}}
\newcommand{\Avn}[1]{{\mathrm{Av}_n\!\left(#1\right)}}
\newcommand{\Co}[1]{{\mathrm{Co}\!\left(#1\right)}}
\newcommand{\Con}[1]{{\mathrm{Co}_n\!\left(#1\right)}}
\newcommand{\st}{\mathrm{st}}
\newcommand{\Gr}[1]{\mathrm{Gr\!\left(#1\right)}}
\newcommand{\Grid}[1]{\mathrm{Grid\!\left(#1\right)}}
\newcommand{\Struct}{\textsf{Struct}}
\newcommand{\CombCov}{\textsf{CombCov}}
\newcommand{\CombSpecSearcher}{\textsf{CombSpecSearcher}}
\newcommand{\boks}[1]{{\lcorners{#1}\rcorners}}
\newcommand{\mediumsqcup}{\textstyle\bigsqcup}
\newcommand{\oeis}[1]{\href{http://oeis.org/#1}{\underline{#1}}}

\theoremstyle{plain}
\newtheorem{theorem}{Theorem}[section]
\newtheorem{proposition}[theorem]{Proposition}
\newtheorem{corollary}[theorem]{Corollary}
\newtheorem{lemma}[theorem]{Lemma}
\newtheorem{conjecture}[theorem]{Conjecture}

\theoremstyle{definition}
\newtheorem{definition}[theorem]{Definition}
\newtheorem{remark}[theorem]{Remark}
\newtheorem{example}[theorem]{Example}

\usepackage{tilings}

%%%%%%%%%%% MACROS FOR DRAWING INTERVAL AND MESH PATTERNS %%%%%%%%%%%

\usepackage{tikz}
\usetikzlibrary{patterns}

% Sub-macros
\newcommand{\shadetheboxesPM}[1]{
    \foreach \x/\y in {#1}
    \fill[pattern color = black!75, pattern=north east lines] (\x,\y) rectangle +(1,1);
}

\newcommand{\drawthegrid}[1]{
    \draw (0.01,0.01) grid (#1+0.99,#1+0.99);
}

\newcommand{\drawverticallines}[3]{
    \foreach \x in {#2}
    \draw[line width=#3] (\x+0.01,0.01) -- (\x+0.01,#1+0.99);
}

\newcommand{\drawhorizontallines}[3]{
    \foreach \y in {#2}
    \draw[line width=#3] (0.01,\y+0.01) -- (#1+0.99,\y+0.01);
}

\newcommand{\drawtheclpattern}[1]{
    \foreach \x/\y in {#1}
    \filldraw (\x,\y) circle (6pt);
}

\newcommand{\drawclpattern}[2]{
	\foreach[count=\x] \y in {#1}
	{
		\filldraw (\x,\y) circle (#2 pt);
	}
}

\newcommand{\drawspecialbox}[1]{
    \foreach \x/\y/\z/\w/\A in {#1}
    {
        \fill[color = white!100, opacity=1, rounded corners = 1.5pt] (\x+0.125,\y+0.125) rectangle (\z-0.125,\w-0.125);
        \draw[color = black, rounded corners = 1.5pt] (\x+0.125,\y+0.125) rectangle (\z-0.125,\w-0.125);
        \fill[black] (\x/2+\z/2,\y/2+\w/2) node {\A};
    }
}

\newcommand{\drawspecialboxlarge}[1]{
    \foreach \x/\y/\z/\w/\A in {#1}
    {
        \fill[color = white!100, opacity=1, rounded corners = 1.5pt] (\x+0.125,\y+0.125) rectangle (\z-0.125,\w-0.125);
        \draw[color = black, rounded corners = 1.5pt] (\x+0.125,\y+0.125) rectangle (\z-0.125,\w-0.125);
        \fill[black] (\x/2+\z/2,\y/2+\w/2) node {\Large \A};
    }
}

\newcommand{\drawsolidshadedbox}[1]{
    \foreach \x/\y/\z/\w/\A in {#1}
    {
        \fill[color = gray!50, opacity=1, rounded corners=1.5pt] (\x+0.125,\y+0.125) rectangle (\z-0.125,\w-0.125);
        \draw[color = black, rounded corners=1.5pt] (\x+0.125,\y+0.125) rectangle (\z-0.125,\w-0.125);
        \fill[black] (\x/2+\z/2,\y/2+\w/2) node {\A};
    }
}

\newcommand{\drawlabels}[1]{
	\foreach \x/\y/\lab in {#1}
	{
		\draw (\x + 0.5,\y + 0.5) node {\lab};
	}
}


\newcommand{\pOneTwo}[1]{\mbox{\patt{#1}{2}{1,2}[][][][][][7]}}
\newcommand{\pTwoOne}[1]{\mbox{\patt{#1}{2}{2,1}[][][][][][7]}}
\newcommand{\pOneTwoThree}[1]{\mbox{\patt{#1}{3}{1,2,3}[][][][][][7]}}
\newcommand{\pOneThreeTwo}[1]{\mbox{\patt{#1}{3}{1,3,2}[][][][][][7]}}
\newcommand{\pTwoOneThree}[1]{\mbox{\patt{#1}{3}{2,1,3}[][][][][][7]}}
\newcommand{\pOneThreeTwoFour}[1]{\mbox{\patt{#1}{4}{1,3,2,4}[][][][][][7]}}

\newcommand{\etcdots}[2]{
	\scalebox{#1}
	{
		\begin{tikzpicture}[baseline=(current bounding box.center)]
			\filldraw (0,2) circle (#2 pt);
			\filldraw (1,1) circle (#2 pt);
			\filldraw (2,0) circle (#2 pt);
		\end{tikzpicture}
	}
}

\newcommand{\etcdotsflipped}[2]{
    \scalebox{#1}
    {
        \begin{tikzpicture}[baseline=(current bounding box.center)]
            \filldraw (0,0) circle (#2 pt);
            \filldraw (1,1) circle (#2 pt);
            \filldraw (2,2) circle (#2 pt);
        \end{tikzpicture}
    }
}

\newcommand{\decr}{\etcdots{0.2}{6}}
\newcommand{\incr}{\etcdotsflipped{0.2}{6}}


% #1: Scale
% #2: Length
% #3: Points
% #4: Shades
% #5: Markings
% #6: Avoidance decorations
% #7: Containment decorations
% #8: Labels
% #9: Size of the points
\newcommand{\patt}[9][4={},5={},6={},7={},8={},9=4]
{
	\scalebox{#1}
	{
		\begin{tikzpicture}[baseline=(current bounding box.center)]
			\useasboundingbox (0.0,-.3) rectangle (#2+1,#2+1.3);
			\shadetheboxesPM{#4}
			\draw (0.01,0.01) grid (#2+1-0.01,#2+1-0.01);

			\drawsolidshadedbox{#6}
			\drawspecialbox{#7}
			\drawspecialboxlarge{#5}
			\drawclpattern{#3}{#9}
			\drawlabels{#8}
		\end{tikzpicture}
	}
}

% #1: Scale
% #2: Length
% #3: Points
% #4: Shades
% #5: Markings
% #6: Avoidance decorations
% #7: Containment decorations
% #8: Circled points
\newcommand{\cpatt}[8][4={},5={},6={},7={},8={}]
{
	\scalebox{#1}
	{
		\begin{tikzpicture}[baseline=(current bounding box.center)]
			\useasboundingbox (0.0,-.3) rectangle (#2+1,#2+1.3);
			\shadetheboxesPM{#4}
			\draw (0.01,0.01) grid (#2+1-0.01,#2+1-0.01);

			\drawsolidshadedbox{#6}
			\drawspecialbox{#7}
			\drawspecialboxlarge{#5}
			\drawclpattern{#3}{4}

			\foreach \x/\y in {#8}
			{
				\draw[line width=1] (\x,\y) circle (7 pt);
			}
		\end{tikzpicture}
	}
}


% #1: Scale
% #2: Width/Height
% #3: Pattern lines
% #4: Added lines
% #5: Points
% #6: Shadings
% #7: Markings
% #8: Ascending restrictions
\newcommand{\metapatt}[8][6={},7={},8={}]
{
    \scalebox{#1}
    {
        \begin{tikzpicture}[baseline=(current bounding box.center)]
					\foreach \width/\height in {#2}
					{
						\useasboundingbox (0.0,-.3) rectangle (\width+1,\height+1.3);
            \shadetheboxesPM{#6}

            \foreach \pos/\type in {#4}
            {
                \ifthenelse{\equal{\type}{v}}
                {
                    \drawverticallines{\height}{\pos}{1.7pt}
                }
                {
								    \ifthenelse{\equal{\type}{d}}
                    {
                      \draw[densely dashed] (\pos,0) -- (\pos,\height+1);
                    }
										{
											\drawhorizontallines{\width}{\pos}{1.7pt}
										}
                }
            }

            \foreach \pos/\type in {#3}
            {
                \ifthenelse{\equal{\type}{v}}
                {
                    \drawverticallines{\height}{\pos}{0.6pt}
                }
                {
										\drawhorizontallines{\width}{\pos}{0.6pt}
                }
            }

            \drawsolidshadedbox{#8}
            \drawspecialbox{#7}

            \foreach \x/\y/\type in {#5}
            {
                \ifthenelse{\equal{\type}{a}}
                {
                    % Added point
                    \draw (\x,\y) circle (6pt);
                    \filldraw (\x,\y) circle (3pt);
                }
                {
                    % Pattern point
                    \filldraw (\x,\y) circle (4pt);
                }
            }
					}
        \end{tikzpicture}
    }
}

% #1: Scale
% #2: Width/Height
% #3: Pattern lines
% #4: Added lines
% #5: Points
% #6: Shadings
% #7: Markings
% #8: Ascending restrictions
% #9: Labels
\newcommand{\dpatt}[9][6={},7={},8={},9={}]
{
    \scalebox{#1}
    {
        \begin{tikzpicture}[baseline=(current bounding box.center)]
					\foreach \width/\height in {#2}
					{
						\useasboundingbox (0.0,-.3) rectangle (\width+1,\height+1.3);
            \shadetheboxesPM{#6}

            \foreach \pos/\type in {#4}
            {
                \ifthenelse{\equal{\type}{v}}
                {
                    \drawverticallines{\height}{\pos}{1.7pt}
                }
                {
								    \ifthenelse{\equal{\type}{d}}
                    {
                      \draw[densely dashed] (\pos,0) -- (\pos,\height+1);
                    }
										{
											\drawhorizontallines{\width}{\pos}{1.7pt}
										}
                }
            }

            \foreach \pos/\type in {#3}
            {
                \ifthenelse{\equal{\type}{v}}
                {
                    \drawverticallines{\height}{\pos}{0.6pt}
                }
                {
										\drawhorizontallines{\width}{\pos}{0.6pt}
                }
            }

            \drawsolidshadedbox{#8}
            \drawspecialbox{#7}

            \foreach \x/\y/\type in {#5}
            {
                \ifthenelse{\equal{\type}{a}}
                {
                    % Added point
                    \draw9 (\x,\y) circle (6pt);
                    \filldraw (\x,\y) circle (3pt);
                }
                {
                    % Pattern point
                    \filldraw (\x,\y) circle (4pt);
                }
            }

						\drawlabels{#9}
					}
        \end{tikzpicture}
    }
}

\newcommand{\mpattern}[4]{										% mesh pattern for text
  \raisebox{0.6ex}{
  \begin{tikzpicture}[scale=0.35, baseline=(current bounding box.center), #1]
  	\useasboundingbox (0.0,-0.1) rectangle (#2+1.4,#2+1.1);

    \shadetheboxesPM{#4}

    \drawthegrid{#2}

    \drawtheclpattern{#3}

  \end{tikzpicture}}
}

% FROM TSA

\newcommand{\textmpattern}[4]{                                      % mesh pattern for text
    \scalebox{#1}
    {
      \begin{tikzpicture}[baseline=(current bounding box.center)]
        \useasboundingbox (0.0,0) rectangle (#2+1,#2+0);
        \shadetheboxesPM{#4}

        \drawthegrid{#2}

        \drawclpattern{#3}{6}

      \end{tikzpicture}
    }
}


% #1: Scale
% #2: Length
% #3: Points
% #4: Shades
% #5: Circled points
% #6: Added lines
% #7: Arrows
\newcommandx{\shpatt}[7][4={},5={},6={},7={}]
{
    \scalebox{#1}
    {
        \begin{tikzpicture}[baseline=(current bounding box.center)]
            \useasboundingbox (0.0,-.3) rectangle (#2+1,#2+1);
            \shadetheboxesPM{#4}
            \draw (0.01,0.01) grid (#2+1-0.01,#2+1-0.01);

            \drawclpattern{#3}{4}

            \foreach \x/\y in {#5}
            {
                \draw[line width=1] (\x,\y) circle (7 pt);
            }

            \foreach \x/\y in {#6}
            {
                \draw[densely dashed, line width=1.7pt] (\x,0) -- (\x,#2+1);
                \draw[densely dashed, line width=1.7pt] (0,\y) -- (#2+1,\y);
            }

            \foreach \xa/\ya/\xb/\yb in {#7}
            {
                \draw[->, line width=1.7pt] (\xa,\ya) -- (\xb-0.12,\yb-0.12);
            }
        \end{tikzpicture}
    }
}

% #1: Scale
% #2: Length
% #3: Points
% #4: Shades
% #5: Circled points
% #6: Added lines
% #7: Arrows
% #8: Special boxes
\newcommandx{\shpattb}[8][4={},5={},6={},7={},8={}]
{
    \scalebox{#1}
    {
        \begin{tikzpicture}[baseline=(current bounding box.center)]
            \useasboundingbox (0.0,-.3) rectangle (#2+1,#2+1);
            \shadetheboxesPM{#4}
            \draw (0.01,0.01) grid (#2+1-0.01,#2+1-0.01);

            \drawclpattern{#3}{4}

            \foreach \x/\y in {#5}
            {
                \draw[line width=1] (\x,\y) circle (7 pt);
            }

            \foreach \x/\y in {#6}
            {
                \draw[densely dashed, line width=1.7pt] (\x,0) -- (\x,#2+1);
                \draw[densely dashed, line width=1.7pt] (0,\y) -- (#2+1,\y);
            }

            \foreach \xa/\ya/\xb/\yb in {#7}
            {
                \draw[->, line width=1.7pt] (\xa,\ya) -- (\xb-0.12,\yb-0.12);
            }
            \foreach \xa/\ya/\xb/\yb in {#8}
            {
                \draw[line width=1.5pt] (\xa+0.1,\ya+0.1) rectangle (\xb-0.1,\yb-0.1);
            }
        \end{tikzpicture}
    }
}

\pgfmathsetmacro{\patttablescale}{1.05}
\pgfmathsetmacro{\pattdispscale}{0.80}
\pgfmathsetmacro{\patttextscale}{0.6}
\pgfmathsetmacro{\patttilingscale}{0.3}


%%%%%%%%%%%%%%% INFORMATION %%%%%%%%%%%%%%%%%%5
%% University information must be multilingual to deal with the
%%  required cover pages and abstract on thesis
%% NOTE: This may not be required for other reports!!!

%% Babel Icelandic macros are setup on RedHat at
%% /usr/share/texlive/texmf-dist/tex/generic/babel/icelandic.sty
%% /usr/share/texlive/texmf-dist/tex/generic/babel-icelandic/icelandic.ldf


%% Multilingual macros
%\newML{macroname}{englishword}{icelandicword}
%  creates \macronameML
%    \MLmacroname[english] - returns the english word
%    \MLmacroname[icelandic] - returns the icelandic word
%    \MLmacroname  - uses the current language setting
% Some useful ones have already been defined, but can be redefined
%% Predefined: \MLIceland \MLReykjavikUniversity \MLUniversityIceland

%% What institute?  Default is RU.
%\setInstitution{\MLReykjavikUniversity}
% \newML{InstitutionAddress}{Menntavegur 1\\101 Reykjavík, Iceland}
% {Menntavegi 1\\101 Reykjavík, Ísland}
% \setInstitutionAddress{\MLInstitutionAddress}
% \newML{Tel}{Tel.}{Sími}
% \setInstitutionPhone{\MLTel{} +354 599 6200\\
% Fax +354 599 6201}
% \setInstitutionURL{www.ru.is}


%% ONLY SET DEPARTMENT IF YOU HAVE NOT USED THE deptsse or deptscs OPTION!
%% Department and degree program
%\newML{ND}{New Department}{Nytt deild}
%\setSchool{\MLND}

%% Set your program of study
\newML{program}{Computer Science}{tölvunarfræði}
\program{\MLprogram}

%% Degree long name.  If not already defined, you can create a macro
%\newML{DEGREE}{English Degree Name}{Icelandic Degree Name}
%% Default is set based upon doctorate vs masters option
%% Predefined: \MLMSc \MLPhd
%\setDegreelong{\MLMSc}

%% Degree abb, change if default is not right
%% Default is set based upon doctorate vs masters option
%\degreeabbrv{Sc.D.} 

%\setFrontLogo{reyst-logo}
%% Use this if you need a different front logo on the first page
%% e.g. reyst-logo

%% Date in english and icelandic
%% NOTE: THIS IS THE DATE OF THE SUPERVISOR'S SIGNATURE!!!!!!
%% Predefined: \MLjan, \MLfeb, \MLmar, ... \MLdec
%\whensigned{day}{month}{year} %day is only used on some formats, but you must put something.
\whensigned{10}{\MLdec}{2019}

%% Title first in English then Icelandic
%% You need to put both a normal case and ALL CAPS version into the macros.
%%
\newML{Title}{Searching for combinatorial covers using linear programming}{Þakningar fléttufræðilegra fyrirbrigða með aðstoð línulegrar bestunnar}
\newML{TITLE}{SEARCHING FOR COMBINATORIAL COVERS USING LINEAR PROGRAMMING}{ÞAKNINGAR FLÉTTUFRÆÐILEGRA FYRIRBRIGÐA MEÐ AÐSTOÐ LÍNULEGRAR BESTUNNAR}
%%
\setTitle{\MLTitle}{\MLTITLE}
%% ***** Special Titles ******
%% If the title must be formatted specifically for the cover page or internal pages
%% (typically via line-breaks using the \newline command) then the following commands must be used 
%%
%\setTitleCover{\MLTITLE}
%% These two for the internal cover pages, usually not needed
%\newML{TitleInternal}{Internal Title}{Icelandic Internal Title}
%\setTitleInternal{\MLTitleInternal}

%% Author name (should be the same in any language, if not use \newML)
%% If you are writing a Project report with multiple authors, separate them with \\:
%% To keep the names typeset together, you want to use non-breaking spaces: ~
%\author{Firstname1~Lastname1\\Firstname2~Lastname2}
\author{Bjarni~Jens~Kristinsson}

%% If the name must be formatted specifically for the signature page
%% (typically via line-breaks) then the following command must be used 
%\setAuthorSignature{Student\\Name}
%% This macro adjusts the author name in the headers of the oldstyle formatting
%\setAuthorHeader{StudentLast}

%% If the thesis is confidential, uncomment this with the date it can be released
%\setClosedDistribution{10.1.2016}%

%% Put your keywords here in English, then Icelandic.  Separate them with commas.
%\newML{keywords}{Keyword1, Keyword2, Keyword3}{Lykliorð1, Lykliorð2, Lykliorð3}
%\setKeywords{\MLkeywords}

%%%%%%%%%%%%%%%%%%%%%%%%%%% Masters Only!! %%%%%%%%%%%%%%%%%%%%%%%%%%%%%%%%%%%%%%%%%%%%
%% How many credits (ECTS) on Master's degree
%% Usually 30 or 60
\ects{30}

%%%%%%%%%%%%%%%%%%%%%%%%%%% Doctorate Only!! %%%%%%%%%%%%%%%%%%%%%%%%%%%%%%%%%%%%%%%%%%
%% Some Computer Science Thesis have an ISSN number.
%% Most other documents do not.
%\bookidnumber{ISSN: 1670-8539} 
%% ID numbers are optional, but nice for sorting in libraries

%% International Standard Book Number (ISBN)
%% This is what most people should use if the thesis is being published.

%% International Standard Serial Number (ISSN)
%% This is usually only for a PhD dissertation as part of a series when published
%%   Computer Science: 1670-8539 

%% Additional degrees?  (optional, usually not needed)
%\adddegree{(list of degrees in appendix)}{(sjá lista yfir prófgraður í viðauka)}
%%%%%%%%%%%%%%%%%%%%%%%%%%%%%%%%%%%%%%%%%%%%%%%%%%%%%%%%%%%%%%%%%%%%%%%%%%%%%%%%%%%%%%%%


%% List the entire committee.  Each member has a name (degree should be omitted, unless it is not PhD),
%% Supervisor(s) must appear first
%% On a Bachelors, there is usually only one supervisor and one examiner.

%% Format for each entry:
%%  \personinfo{Name}{Role}{Job Title}{Company/institution}{Country}
%% Predefined macros: \MLSupervisor \MLSupervisors \MLExaminer \MLExaminers

%% Change these to singular/plural as needed.
%% Just uncomment and change the plurality of the macro.
%\setSupervisorHeading{\MLSupervisors}
%\setExaminerHeading{\MLExaminer}

%% Predefined macros:
%% \MLSeniorProfessor \MLProfessor \MLAssociateProfessor \MLAdjunctProfessor \MLEmeritusProfessor \Iceland
%% \MLReykjavikUniversity \MLUniversityIceland

%% All others: As many as you want
\supervisors{
  \personinfo{Henning A. Ulfarsson}{\MLSupervisor}{\MLAssistantProfessor}{Reykjavik University}{\MLIceland}
  \personinfo{Christian Bean}{\MLSupervisor}{Postdoctoral Researcher}{Reykjavik University}{\MLIceland}
}

%% All others: As many as you want
\examiners{
  \personinfo{Anders Claesson}{\MLExaminer}{\MLProfessor}{\MLUniversityIceland}{\MLIceland}
  \personinfo{Bjarki Ágúst Guðmundsson}{\MLExaminer}{Software Engineer}{Google}{Switzerland}
}

%% An abstract is required to be in both Icelandic and English for most degrees.
%% It is considered good form to limit the abstract to a single paragraph in each language,
%%   at 300 words.  Refer to your degree's instructions.
%% Note: Icelandic quotation marks cannot be typeset using "` and "'.  You should use \enquote{}
%% this is probably due to interactions with the MultiLingual macros.
\newML{AbstractText}{
  We introduce the \CombCov\ framework which is a generalization of the \Struct\ algorithm introduced by \citeauthor{bean_automatic_2019} in \emph{\citetitle{bean_automatic_2019}}. We give a simple example of an application of the framework to \emph{avoidance sets} of \emph{words} and discuss in detail how to generate \emph{rules} of lesser complexity and how a cover is verified up to a certain size using linear programming. We then apply the framework to various published results on \emph{permutations} avoiding \emph{mesh patterns} and try to find covers of similar problems with some success. We show that \CombCov\ is a powerful tool in guiding humans by coming up with conjectures that would otherwise have required substantial effort to discover manually.
}
{
  Við kynnum hugbúnaðinn \CombCov\ sem er útvíkkun á \Struct\ reikniritinu eftir Christian, Bjarka og Henning úr \emph{\citetitle{bean_automatic_2019}}. Við skoðum einfalt dæmi um notkun þess með \emph{forðunarmengjum} af \emph{orðum} og skýrum hvernig \emph{reglur} eru búnar til og hvernig þakning er sannreynd upp að ákveðinni lengd með hjálp línulegrar bestunnar. Síðan beitum við hugbúnaðinum á ýmis þekkt vandamál á sviði \emph{umraðana} sem forðast \emph{möskvamynstur} og reynum einnig að finna lausnir á áður óleystum vandamálum með einhverjum árangri. Þannig sýnum við að \CombCov\ er öflugt tól sem getur aðstoðað mannfólk við að fá hugmyndir að lausnum sem því hefði annars kannski ekki dottið í hug nema með mikilli fyrirhöfn.
} % Icelandic abstract goes here
\setAbstract{\MLAbstractText}


%%%%%%%%%%%%%%INDEX SETUP %%%%%%%%%%%%%%%%%%%%%%%%%%%%%%%%%%%%%%%%%%%%%%%%%%%%
%% Indexes, and other auto-generated material
%% The Memoir package (which we use) automatically generates the index
%% See section 17.2 on page 302 of the guide
%% http://texdoc.net/texmf-dist/doc/latex/memoir/memman.pdf
%% This means you have to run "makeindex DEGREE-NAME-YEAR"
%% !!!Do not load any of the index packages, they cause problems with Memoir!!!
%% !!!You have been warned!!!committe
%% Note that memoir changes the [] options to only be for filenames, not other options!
\makeindex{}
\indexintoc{}

%% For abbreviations, you may want to try
%% Watch out though, each new index writes another external file and 
%% latex can only write a limited number of them
%%\usepackage[intoc]{nomencl} % intoc: In Table of Contents
%% remember to run:
%% makeindex filename.nlo  -s nomencl.ist -o filename.nls

\finalifforcegraphics{hyperref} %hyperlinks even in draft mode
\usepackage[hidelinks]{hyperref} 
%% !!!Must be the last package loaded except otherwise mentioned!!!!
\usepackage{hypcap}  %% puts link at top of figure, must be after hyperref

%%%%%%%%%%%%%%%%%%%%%%%%%%%%%%%%%%%%%%%%%%%%%%%%%%%%%%%%%%%%%%%%%%%%%%%%%%
%%%%%%%%%%%%%%%%%%%%%%% DOCUMENT START %%%%%%%%%%%%%%%%%%%%%%%%%%%%%%%%%%%
\begin{document}
%% Some elements have different names on the RU Masters rules
%% They will be annotated with RUM: "name"
\frontmatter{} % setup formatting at beginning

%\frontcover{}%%If you want to see what it looks like with the printed cover

\frontrequiredpages{}%% the various signaturepages and abstract
%%% WARNING:  if you get an error on the previous line, it is probably because
%%% you put a bad macro or something strange in a title, author, or abstract.

%% Dedication is optional, comment out if it is absent
%% RUM: Not mentioned
% \begin{dedications}
%   I dedicate this to Greta Thunberg,\\ for being the hero the world needs.
% \end{dedications}

\enableindents{}% turn on/off paragraph indents
% RUM: "Acknowledgements (optional)"
% \coverchapter{Acknowledgments} 
% \begin{quotation}
% I would like to thank people and machines.
% \end{quotation}

% \coverchapter{Preface}
% % RUM: "Preface (optional)"
% This dissertation is original work by the author, Firstname~Lastname.
% Portions of the introductory text are used with permission from
% Student et al.\cite{student2015awesome} of which I am an author.

%\coverchapter{Publications}
%% RUM: Not mentioned, this was found in the CS thesis template.  
%% Maybe more applicable to PhD dissertations?
%%% Probably a duplication from before Preface became standard.

\starttables{}% setup formatting
%% TOC, list of figures and list of tables are required
\tableofcontents{}\clearpage%%RUM: "Table of contents"
\listoffigures{}\clearpage%%RUM: "List of figures"
\listoftables{}\clearpage%%RUM: "List of tables"

%% This command prepares for the actual text, e.g. by 
%% calling \mainmatter{}
\starttext{}

%% ---------------------------------------------------------------
%% From this point on, it is standard Latex, except the very end.
%% This is a "report"-based template, so the top-level heading 
%% is \chapter{}

%% WARNING: Make sure that all of these files (and any new ones)
%% are UTF-8 otherwise you will get weird encoding errors.

%% The default division is IMRAD, you may want to divide differently
%% See the introduction for guidance.

\chapter{Introduction\label{cha:introduction}}

In the last few years we have seen increased interest in \emph{experimental 
mathematics} where computers are used to perform experiments and analyze data 
in greater quantities than humans are able to process by hand. Combinatorics is 
one field of mathematics that has greatly benefited from the advent of 
computers. Researchers routinely write computer programs that can quickly 
verify or refute hypotheses made by themselves to enable them to more
efficiently allocate their time and energy with regards to the direction of the 
research. 

Even more interesting are computer programs that can act as a source of
inspiration by finding obscure and hidden patterns in large problem spaces. 
This can help the human researcher by giving them the correct idea to further
pursue. One such tool is \Struct\ by \textcite{bean_automatic_2019} which can
automatically discover the structural rules of \emph{permutation 
classes}\footnote{We will define this and other terminology in 
Section~\ref{mesh tilings background}.} and thus conjecture the 
\emph{enumeration} of the set as a function of the length of the 
\emph{permutations} in the set. This is a problem whose solution can be hard to 
discover but easy to prove once found.

Other more complex computer programs exist but they often come with a drawback 
or compromise between usability and effectiveness. \CombSpecSearcher\ by 
\textcite{bean_finding_2018} can find the enumeration of permutation classes, 
along with a proof of its validity, but it required significant work on 
behalf of the researcher in implementing \emph{strategies} specific to the 
domain of permutations. Applying it to another domain would require similar
effort.  

\Struct\ and our extended version \CombCov\ have a lower barrier of entry for 
users in the hope that it will be more widely adopted by researchers. The lack 
of computer generated proofs are mitigated by the fact that the conjectures are 
often easily verified or disproved by the user.


%%%%%%%%%%%%%%%%%%%%%%%%%%%%%%%%%%%%%%%%%%%%%%%%%%%%%%%%%%%%%%%%%%%%%%%%%%%%%%%%
\section{Organization of thesis}

We dedicate Chapter~\ref{words chapter} to explaining how \CombCov\ works from a 
high-level perspective. In Section~\ref{words background} we define \emph{words}
as an intuitive example of \emph{avoidance sets} and in Section~\ref{words 
implementation} we explain how we implement them with \CombCov. In 
Section~\ref{words results} we present some results and analyze the 
effectiveness of \CombCov\ as a tool.

Chapter~\ref{mesh tilings chapter} concerns itself with our main application of 
\CombCov. In Section~\ref{mesh tilings background} we define \emph{permutations}
and \emph{permutation classes}, the domain in which \Struct\ operates, and 
\emph{mesh patterns} and \emph{mesh tilings}. Our implementation of mesh tilings
with  the framework is discussed in Section~\ref{mesh tilings implementation} 
and the results are presented in Section~\ref{mesh tilings results}.

Finally, Chapter~\ref{discussion chapter} concludes the thesis and discusses 
future work on this topic.
%%RUM: Introduction
\chapter{Words\label{words chapter}}

This chapter aims to explain the algorithm behind \CombCov\ using avoidance sets
of words as an intuitive example. Some results from this is discussed in 
Section~\ref{words results} giving insight into the limitations of \CombCov\ as 
a tool.

%%%%%%%%%%%%%%%%%%%%%%%%%%%%%%%%%%%%%%%%%%%%%%%%%%%%%%%%%%%%%%%%%%%%%%%%%%%%%%%%
\section{Background\label{words background}}

\textcite{bean_automatic_2019} implemented the \Struct\ algorithm in Python and 
published it on GitHub \cite{bean_permstruct_2017}. The core idea is to write a 
permutation class $\Av{\Pi}$ as a disjoint set of ``simpler'' sets of 
permutations $S_i$ so that \[\bigcup_i S_i = \Av{\Pi}\] and $S_i \cap S_j = 
\emptyset$ for $i \neq j$. The challenge is to come up with these simpler 
subsets and verify that the two conditions hold (union and disjointedness). In 
this context we call $\Av{\Pi}$ the \emph{root object} and $S_i$ the 
\emph{rules} or \emph{subrules}. \CombCov\ aims to be a general framework to 
solve these problems for any kind of combinatorial object. It is available as a 
Python module\footnote{Installable via \texttt{pip install CombCov}} (with 
source code hosted on GitHub \cite{kristinsson_combcov:_2019}) and only requires 
the user to come up with, and implement in Python, how to generate the rules. 
The rest is handled by the framework.

\begin{definition}
  A \emph{word of length $n$} is a sequence of \emph{characters} $c_1 \cdots 
  c_n$ over an \emph{alphabet} $\Sigma$. If $n = 0$ then the word is the 
  \emph{empty word} and we denote it with $\epsilon$.
\end{definition}

In what follows of this thesis we only concern ourselves with words over the two 
letter alphabet $\Sigma = \left\{ a,b \right\}$ and the reader should assume 
this choice of alphabet if it not specifically stated. An example of a word of 
length 4 is $abba$.

\begin{definition}
  We say that a word $u = u_1 \cdots u_n$ \emph{contains} another word $v = v_1 
  \cdots v_k$ as a \emph{subword} if there exists an $i$ such that $u_{i+1} 
  \cdots u_{i+k} = v_1 \cdots v_k$. If $u$ does not contain $v$, we say that $u$ 
  \emph{avoids} $v$ and define $\Avn{v}$ as the set of all words of length $n$ 
  avoiding $v$ and write $\Av{v} = \bigcup\limits_{n=0}^{\infty}\Avn{v}$.
\end{definition}

\begin{example}
  The word $abba$ contains the subword $bb$ but avoids $aa$.
\end{example}

\begin{definition}
  The \emph{concatenation} of two words $u = u_1 \cdots u_m$ and $v = v_1 \cdots 
  v_n$ is the word $uv = u_1 \cdots u_m v_1 \cdots v_n$ of length $m+ n$. The 
  concatenation of a word $u$ and a set $\mc{S}$ of words is written either 
  $u\mc{S}$ or $u + \mc{S}$ and defined as the set $\{ us \colon s \in 
  \mc{S}\}$. In this context we say that $u$ is the \emph{prefix} of $uv$ and 
  $u\mc{S}$.
\end{definition}

For a set of words $\mc{Z}$ over an alphabet $\Sigma$ we define the 
\emph{avoidance set of words} $\Av{\mc{Z}}$ as the set of words that avoid all 
words in $\mc{Z}$. The avoidance set is \emph{closed downwards}, meaning that if 
$W$ is a word in the avoidance set then all subwords $w$ of $W$ are also in the 
avoidance set. The easiest way to see this is by contradiction: If for a word 
$W \in \Av{\mc{Z}}$ there exists a subword $w$ of $W$ such that $w \notin 
\Av{\mc{Z}}$ then there exist a $z \in \mc{Z}$ such that $z$ is contained in 
$w$, and because $w$ is a consecutive subsequence in $W$ then $z$ is contained 
in $W$ meaning $W \notin \Av{\mc{Z}}$, contradicting the initial assumption.


%%%%%%%%%%%%%%%%%%%%%%%%%%%%%%%%%%%%%%%%%%%%%%%%%%%%%%%%%%%%%%%%%%%%%%%%%%%%%%%%
\section{Implementation\label{words implementation}}

We now discuss how we implement the abstract idea of words with \CombCov\ and
go step by step through the algorithm how the framework searches for covers of 
these avoidance sets.

%%%%%%%%%%%%%%%%%%%%%%%%%%%%%%%%%%%%%%%%%%%%%%%%%%%%%%%%%%%%%%%%%%%%%%%%%%%%%%%%
\subsection{Computing with finite sets\label{computing with finite sets}}

Recall that for any set $S$ of words we can write $\Av{S} = 
\bigcup\limits_{n=0}^{\infty}\Avn{S}$. \CombCov\ takes in a parameter 
\texttt{max\_elmnt\_size}\footnote{In Section~\ref{words results} we set 
\texttt{max\_elmnt\_size = 7}.} and considers only \[ R = 
\bigcup\limits_{n=0}^{\texttt{max\_elmnt\_size}}\Avn{S} \] for it computations. 
The hope is that the conclusions \CombCov\ makes for $R$ generalizes to all 
element sizes, something that is often easy for a human to verify (or disprove).

Now assume $R = \left\{ x_1, \ldots, x_m \right\}$. For a subset $R'$ of $R$ we 
define the \emph{binary containment string} (or simply \emph{containment 
string}) of $R'$ to be the $m$-long binary string $B' = b_1 \cdots b_m$ where 
$b_i$ is equal to 1 if $x_i \in R'$ and 0 otherwise. We assume a consistent 
order for the elements of $R$ and in the case of words we use lexicographical 
ordering.

E.g., if $R = \{\epsilon, a, b, ab, ba, bb\}$ then $B' = 111111$ denotes the 
whole set $R$ and $B'' = 011001$ denotes the subset $\{a, b, bb\}$ and $B''' = 
100000$ the subset $\{\epsilon\}$.


%%%%%%%%%%%%%%%%%%%%%%%%%%%%%%%%%%%%%%%%%%%%%%%%%%%%%%%%%%%%%%%%%%%%%%%%%%%%%%%%
\subsection{Disjoint subsets}

We want to find a disjoint set of non-empty subsets $R_i$ of $R$ that covers 
$R$, i.e., \[\bigcup_i R_i = R\] with $R_i \cap R_j = \emptyset$ for $i \neq j$. 
Just as $R$ is a finite representation of the (possibly) infinite $\Av{S}$ (the 
root object), each $R_i$ should be a finite version of a (possibly) infinite set 
$S_i$ (the rules). The hope is that the finite cover generalizes to the 
(possibly infinite) cover \[\bigcup_i S_i = \Av{S}\] with $S_i \cap S_j = 
\emptyset$ for $i \neq j$. We summarize this in Table~\ref{table:ComCov infinite 
finite abstraction}.

\begin{table}[ht]
    \centering
    \begin{tabular}{ | r || c | c | }
        \hline
         & (Possibly) infinite set & Finite representation \\
        \hline\hline
        Root object & $\Av{S}$ & $R$   \\ \hline
        Rules       & $S_i$    & $R_i$ \\ \hline
    \end{tabular}
    \caption{\CombCov\ uses finite representations of the root object and the rules}
    \label{table:ComCov infinite finite abstraction}
\end{table}


%%%%%%%%%%%%%%%%%%%%%%%%%%%%%%%%%%%%%%%%%%%%%%%%%%%%%%%%%%%%%%%%%%%%%%%%%%%%%%%%
\subsection{Generating the subsets}

As previously mentioned, the way we generate these simpler subset cannot be 
abstracted away in \CombCov\ as we need to do it differently with different 
combinatorial objects. The rules are \emph{descriptions} of the subsets $S_i$ 
of $\Av{S}$ and the finite counterparts $R_i$ of $R$.

This is the part where the user can apply their expertise to come up with a way 
of generating these rules and be  smart in how they are doing it to get the best 
possible conjectures from \CombCov. The framework can only check if the $R_i$'s 
are proper subsets of $R$, but not if the $S_i$'s are proper subsets of 
$\Av{S}$. The user should generate rules that are likely to generalize well but 
increasing the \texttt{max\_elmnt\_size} also increases the chance of that 
happening (at the cost of computing time). Wrong covers are often easily spotted 
when trying to prove (and then disprove) the conjectures.

Assuming $S = \{ s_1, \ldots, s_n \}$ is a set of $n$ words, the longest of 
length $k$, we decided to create rules of the form $S_i = u \Av{S'}$ where $u$ 
is a word in $\Av{S}$ of length at most $k$ chosen from $R$ and $S' = \Sigma$ 
(the whole alphabet) or $S'$ is a set of words each of which is a subword of a 
word in $S$. In addition we sort $S'$ by lexicographical order and check for 
every word $s' \in S'$ if it is contained in a longer word $s'' \in S'$, and if 
so, remove $s''$ from $S'$.

% For an avoidance set of words $\Av{S}$ we decided to create rules of the form 
% $S_i = u \Av{S'}$ defined as follows:

% \begin{itemize}
%     \item $S = \{ s_1, \ldots, s_n \}$ is a set of $n$ words. Let $k$ be the 
%     	length of the biggest word in $S$.
%     \item $u$ is a word in $\Av{S}$ of length at most $k$ chosen from $R$
%     \item $S' = \Sigma$ (the whole alphabet) or $S'$ is a set of words each of 
%     	which is a subword of a word in $S$. 
%     \item Sort $S'$ by lexicographical order and check for every word 
%     	$s' \in S'$ if it is contained in a longer word $s'' \in S'$, and if so, 
%     	remove $s''$ from $S'$.
% \end{itemize}

\CombCov\ now generates the finite sets $R_i$ of all elements in $S_i$ of size 
up to \texttt{max\_elmnt\_size}. If the same element is generated in more than 
one way the rule is discarded as \emph{invalid}. This does not happen with our 
subrules, but a rule like $\Av{a} + \Av{a}$ would be invalid because it 
generates $bbb$ in multiple ways.

After this the framework checks each $R_i$ is a proper subset of $R$ and if so 
the corresponding rule $S_i$ is said to be \emph{valid}. \CombCov\ discards all 
invalid rules. To optimize performance, the user should avoid creating too many 
invalid rules.


%%%%%%%%%%%%%%%%%%%%%%%%%%%%%%%%%%%%%%%%%%%%%%%%%%%%%%%%%%%%%%%%%%%%%%%%%%%%%%%%
\subsection{Finding a cover with the rules}

For each of the valid rules \CombCov\ constructs the corresponding binary 
containment string $B^{(i)}$, e.g., with \texttt{max\_elmnt\_size = 2} and $R$ 
as in Section~\ref{computing with finite sets} the rule $a \Av{a}$ generates 
the set $R_1 = \{ a, ab \} \subseteq R$ with the corresponding containment 
string $010100$. The rule $a \Av{b}$ generates the set $R_2 = \{ a, aa \} \not 
\subseteq R$ and is thus invalid.

Without loss of generality, assume that the $k$ valid rules are the first $k$ 
ones $S_1, \ldots, S_k$. The sets $R_1, \ldots, R_k$ with corresponding 
containment strings $B^{(1)}, \ldots, B^{(k)}$ are not necessarily disjoint and 
some may even be equal to each other. \CombCov\ constructs a \emph{linear 
problem} of $k$ Boolean variables with $m$ equations minimizing the sum of the
variables:

\begin{alignat*}{5}
    \text{Min}  \qquad  & \mathrlap{z = x_1 + \dotsb + x_k}   & & & & & & & \\
    \text{s.t.} \qquad  & &                             x_1 b^{(1)}_{1}         & +{} & \dotsb & +{} &                             x_k b^{(k)}_{1}             & ={} & 1      \\
                        & & \mathrel{\makebox[\widthof{$x_1 b^{(1)}_{j}$}]{\vdots}} & & \ddots &     & \mathrel{\makebox[\widthof{$x_1 b^{(k)}_{j}$}]{\vdots}} & ={} & \vdots \\
                        & &                             x_1 b^{(1)}_{m}         & +{} & \dotsb & +{} &                             x_k b^{(k)}_{m}             & ={} & 1      \\
    \text{with} \qquad  & \mathrlap{x_i \in \left\{ 0, 1 \right\} \text{ for } i = 1, \dotsc, k.} & & & & & & &
\end{alignat*}

The variables $x_i$ denotes whether rule $i$ is part of the cover or not, 
$b^{(j)}_{i}$ the Boolean value representing if element $j$ of $R$ is in the 
finite set $R_i$. \CombCov\ uses the linear problem solver Gurobi 
\cite{lcc_gurobi_optimization_gurobi_2019} (with fallback on COIN CLP/CBC LP 
\cite{coin-or_coin_2019}) to return a subset $I \subseteq \llbracket k 
\rrbracket$ with indices of the rules constituting the cover, if a cover is 
found. Note that the linear problem is defined as a minimization problem because 
we want solutions consisting of as few rules as possible.


%%%%%%%%%%%%%%%%%%%%%%%%%%%%%%%%%%%%%%%%%%%%%%%%%%%%%%%%%%%%%%%%%%%%%%%%%%%%%%%%
\subsection{An example with $\Av{aa}$}

In Table~\ref{table:subrules of Av(aa)} we list the rules, corresponding subsets 
$R_i$ and binary containment strings for the avoidance set $\Av{aa}$ with 
\texttt{max\_elmnt\_size = 2}. Those are the subrules of the root object 
$\Av{aa}$.

{\renewcommand{\arraystretch}{1.5}
\begin{table}[ht]
    \centering
    \begin{tabular}{ c | c | c }
        Rule & $R_i$ & Containment string \\
        \hline\hline
        $\epsilon \Av{a,b}$ & $\{ \epsilon \}$ & $100000$ \\
        $a \Av{a,b}$ & $\{ a \}$ & $010000$ \\
        $b \Av{a,b}$ & $\{ b \}$ & $001000$ \\
        $ab \Av{a,b}$ & $\{ ab \}$ & $000100$ \\
        $ba \Av{a,b}$ & $\{ ba \}$ & $000010$ \\
        $bb \Av{a,b}$ & $\{ bb \}$ & $000001$ \\
        $a \Av{a}$ & $\{ a, ab \}$ & $010100$ \\
        $a \Av{aa}$ & $\{ a, aa, ab \}$ & \emph{invalid} \\
        $b \Av{a}$ & $\{ b, bb \}$ & $001001$ \\
        $b \Av{aa}$ & $\{ b, ba, bb \}$ & $001011$ \\
        $ab \Av{a}$ & $\{ ab \}$ & $000100$ \\
        $ab \Av{aa}$ & $\{ ab \}$ & $000100$ \\
        $ba \Av{a}$ & $\{ ba \}$ & $000010$ \\
        $ba \Av{aa}$ & $\{ ba \}$ & $000010$ \\
        $bb \Av{a}$ & $\{ bb \}$ & $000001$ \\
        $bb \Av{aa}$ & $\{ bb \}$ & $000001$ \\
    \hline \hline
    \end{tabular}
    \caption{Subrules of $\Av{aa}$ over the alphabet $\Sigma = \{a, b\}$}
    \label{table:subrules of Av(aa)}
\end{table}}

Note that no two rules in the table are the same but because of the low value 
of \texttt{max\_elmnt\_size} some of the subsets $R_i$ (the containment strings) 
are the same. This would not happen with higher values of 
\texttt{max\_elmnt\_size} as the rules are truly different and would start 
generating different words, as we see in Section~\ref{results:Av(aa)}. Running 
this exact problem in \CombCov\ gives the 3-rule solution \[ \Av{aa} = 
\epsilon \Av{a,b} \cup a \Av{a} \cup b \Av{aa} \] with corresponding bitstrings 
$100000$, $010100$ and $001011$. This is certainly a correct cover of $R$, but 
\emph{not} of $\Av{aa}$. It's easy to see that $abba \in \Av{aa}$ but none of 
the rules is able to generate this word. After seeing this, and rerunning 
\CombCov\ with sufficiently high value of \texttt{max\_elmnt\_size} we 
eventually get the correct solution \[\Av{aa} = \epsilon \Av{a,b} \cup a 
\Av{a,b} \cup b \Av{aa} \cup ab \Av{aa}\] as seen in Section~\ref{words results} 
where \texttt{max\_elmnt\_size = 7}.


%%%%%%%%%%%%%%%%%%%%%%%%%%%%%%%%%%%%%%%%%%%%%%%%%%%%%%%%%%%%%%%%%%%%%%%%%%%%%%%%
\section{Results\label{words results}}

All results presented in this section as well as in Section~\ref{mesh tilings 
results} were obtained using version \texttt{v0.6.3} of \CombCov, which is the 
latest version at the time of writing, and \texttt{max\_elmnt\_size = 7}. In 
this section only the results were obtained by running the software on a 2017 
model MacBook Pro laptop with execution times on the scale of seconds to 
minutes. 

\subsection{Overview of results}

We applied \CombCov\ on a select few avoidance sets of words over the alphabet 
$\Sigma = \{a,b\}$. The avoiding subword sets were selected with one or two 
words, each of length at most three. \CombCov\ prints out the enumerations of 
the sequences up to length \texttt{max\_elmnt\_size} which we include along with 
the corresponding OEIS \cite{oeis_foundation_inc._-line_2019} sequence number. 
The results are presented in Table~\ref{table:some avoidance sets of words}.

{\renewcommand{\arraystretch}{1.5}
\begin{table}[ht]
    \centering
    \begin{tabular}{ c | c | l | l }
        Avoiding subwords & Cover Found & Enumeration & OEIS \\
        \hline\hline
        $\emptyset$ & \emph{No} & $1, 2, 4, 8, 16, 32, 64, 128$ & \oeis{A000079} \\ \hline
        $\{a\}$ & 
            $\epsilon \Av{a,b} \cup b \Av{a}$ & 
            $1, 1, 1, 1, 1, 1, 1, 1$ & \oeis{A000012} \\ 
        $\{a, b\}$ & 
            $\epsilon \Av{a,b}$ & 
            $1, 0, 0, 0, 0, 0, 0, 0$ & \oeis{A000007} \\ \hline
        $\{aa\}$ & 
            {\renewcommand{\arraystretch}{1}
            \begin{tabular}{@{}c@{}}
                $\epsilon \Av{a,b} \cup a \Av{a,b}$ \\ 
                $\cup b \Av{aa} \cup ab \Av{aa}$
            \end{tabular}} & 
            $1, 2, 3, 5, 8, 13, 21, 34$ & \oeis{A000045} (shifted) \\ 
        $\{aa, b\}$ & 
            $\epsilon \Av{a,b} \cup a \Av{a,b}$ & 
            $1, 1, 0, 0, 0, 0, 0, 0$ & \oeis{A019590} \\ 
        $\{aa, bb\}$ & 
            \emph{No} & 
            $1, 2, 2, 2, 2, 2, 2, 2$ & \oeis{A040000} \\ \hline
        $\{ab\}$ & 
            $\epsilon \Av{a,b} \cup a \Av{b} \cup b \Av{ab}$ & 
            $1, 2, 3, 4, 5, 6, 7, 8$ & \oeis{A000027} \\ 
        $\{ab, ba\}$ & 
            $\epsilon Av(a,b)\cup a \Av{b} \cup b \Av{a}$ & 
            $1, 2, 2, 2, 2, 2, 2, 2$ & \oeis{A040000} \\ \hline
        $\{aaa\}$ & 
            {\renewcommand{\arraystretch}{1}
            \begin{tabular}{@{}c@{}c@{}}
                $\epsilon \Av{a,b} \cup a \Av{a,b}$ \\
                $\cup aa \Av{a,b} \cup b \Av{aaa}$ \\
                $\cup ab \Av{aaa} \cup aab \Av{aaa}$ & 
            \end{tabular}} & 
            $1, 2, 4, 7, 13, 24, 44, 81$ & \oeis{A000073} (shifted) \\ 
        $\{aaa, b\}$ & 
            $\epsilon \Av{a,b} \cup a \Av{aa,b}$ & 
            $1, 1, 1, 0, 0, 0, 0, 0$ & \oeis{A130716} \\ 
        $\{aaa, bb\}$ & 
            \emph{No} & 
            $1, 2, 3, 4, 5, 7, 9, 12$ & \oeis{A164001} \\ 
        $\{aaa, bbb\}$ & 
            \emph{No} & 
            $1, 2, 4, 6, 10, 16, 26, 42$ & \oeis{A128588} \\ \hline 
        $\{aba\}$ & 
            \emph{No} & 
            $1, 2, 4, 7, 12, 21, 37, 65$ & \oeis{A005251} (shifted) \\ 
        $\{aba, aa\}$ & 
            {\renewcommand{\arraystretch}{1}
            \begin{tabular}{@{}c@{}c@{}}
                $\epsilon \Av{a,b} \cup a \Av{a,b}$ \\
                $\cup b \Av{aa,aba} \cup ab \Av{a,b}$ \\
                $\cup abb \Av{aa,aba}$ &
            \end{tabular}} & 
            $1, 2, 3, 4, 6, 9, 13, 19$ & \oeis{A000930} (shifted) \\ 
        $\{aba, bb\}$ & 
            {\renewcommand{\arraystretch}{1}
            \begin{tabular}{@{}c@{}}
                $\epsilon \Av{a,b}\cup a \Av{ba,bb}$ \\
                $\cup b \Av{a,b} \cup ba \Av{ba,bb}$ 
            \end{tabular}} & 
            $1, 2, 3, 4, 4, 4, 4, 4$ & \oeis{A158411} (shifted) \\ 
        $\{aba, bab\}$ & 
            \emph{No} & 
            $1, 2, 4, 6, 10, 16, 26, 42$ & \oeis{A128588} \\ 
        \hline \hline
    \end{tabular}
    \caption{Some avoidance sets of words over the alphabet $\Sigma = \{a, b\}$}
    \label{table:some avoidance sets of words}
\end{table}}

We are encouraged to see that \CombCov\ manages to find covers for this many 
examples. We now take a closer look at some of the results.


%%%%%%%%%%%%%%%%%%%%%%%%%%%%%%%%%%%%%%%%%%%%%%%%%%%%%%%%%%%%%%%%%%%%%%%%%%%%%%%%
\subsection{The avoidance set $\Av{aa}$\label{results:Av(aa)}}

The algorithm suggests that \[\Av{aa} = \epsilon \Av{a,b} \cup a \Av{a,b} \cup b 
\Av{aa} \cup ab \Av{aa}\] where on the right-hand side there are four disjoint 
subsets, verified by the algorithm for all elements size up to 7. The cover is 
indeed correct, as can be seen by thinking of a word $w \in \Av{aa}$ in a series 
of \emph{if-else} statements of the first few characters and using recursion. 
We will now show that the cover yield the sequence of the (shifted) Fibonacci 
numbers.

Recall that $\epsilon$ is the empty word and note that $\Av{a,b}$ avoids all 
words that contains either $a$ or $b$ so $\epsilon \Av{a,b} = \{ \epsilon \}$ 
and $a \Av{a,b} = \{ a \}$. Now assume $c_n$ is the number of words in $\Av{aa}$ 
of length $n$ and write $\sum_{n \geq 0} c_n x^n$ as the generating function for 
$\left| \Avn{aa} \right|$. Then the cover gives us that \[\sum_{n \geq 0} c_n 
x^n = 1 + x + \sum_{n \geq 0} c_n x^{n+1} + \sum_{n \geq 0} c_n x^{n+2} .\] By 
looking at the coefficients at $x^0$ we see that $c_0 = 1$ and by comparing the 
coefficients at $x^1$ we get that $c_1 = 1 + c_0$ i.e., $c_1 = 2$. For $n \geq 
2$ we get the recurrence relation \[ c_{n} = c_{n-1} + c_{n-2} \] which proves 
the enumeration.


%%%%%%%%%%%%%%%%%%%%%%%%%%%%%%%%%%%%%%%%%%%%%%%%%%%%%%%%%%%%%%%%%%%%%%%%%%%%%%%%
\subsection{The fault lies with the user}

It is clear that our implementation of subrule generation for avoidance sets of 
words, the format of $u \Av{S'}$, is not general enough to find a cover for all 
avoidance sets of words. It is not the fault of the tool, but of the one who 
yields it.

It is interesting to see that $\Av{aaa, bbb}$ and $\Av{aba, bab}$ both have the 
same enumeration (up to length 7) but \CombCov\ is unable to find a cover for 
either of the avoidance sets of words. Meanwhile, $\Av{aa, bb}$ and 
$\Av{ab, ab}$ also have the same enumeration of which \CombCov\ finds a cover 
for the latter but not the former. It is simple to confirm that \[\Av{ab, ba} = 
\epsilon Av(a,b)\cup a \Av{b} \cup b \Av{a}\] as the first rule is the set of 
the empty word, the second consists of all non-empty words of only $a$'s and the 
third rule generates all non-empty words consisting of only $b$'s. It is a 
different description of the set of words that contains neither $ab$ or $ba$.

\chapter{Mesh tilings\label{mesh tilings chapter}}

In this chapter we review the terminology on permutations and mesh patterns 
before defining \emph{mesh tilings}. In Section~\ref{mesh tilings 
implementation} we explain how we apply \CombCov\ to mesh tilings and discuss 
the results in Section~\ref{mesh tilings results}. We consider this to be our
main contribution apart from the framework itself.

%%%%%%%%%%%%%%%%%%%%%%%%%%%%%%%%%%%%%%%%%%%%%%%%%%%%%%%%%%%%%%%%%%%%%%%%%%%%%%%%
\section{Background\label{mesh tilings background}}

\begin{definition}
  A \emph{(classical) permutation of length $n$} is a bijection from the set of 
  the first $n$ integers, $\llbracket n \rrbracket = \left\{ 1, 2, \ldots, n 
  \right\}$ to itself. The set of all permutations of length $n$ is denoted with 
  $\S_n$ and $\S = \bigcup\limits_{n=0}^{\infty}\S_n$ is the set of all 
  permutations.
\end{definition}

A permutation can also be viewed as a string $\pi(1)\cdots\pi(n)$ by listing the 
values that $\pi$ takes at each index. An example of this is $\pi = 35142$, a 
permutation of length five with $\pi(1) = 3$, $\pi(2) = 5$, $\pi(3) = 1$, 
$\pi(4) = 4$ and $\pi(5) = 2$. The visual \emph{grid representation} of $\pi$, 
denoted with $\Gr{\pi}$, is the plot of $\left\{ (i, \pi(i)) \mid i \in 
\llbracket n \rrbracket \right\}$ in a Cartesian coordinate system, shown in 
Figure~\ref{figure: Gr(35142)}. The only permutation of length 0 is called the 
\emph{empty permutation} and it is denoted by $\epsilon$.

\begin{figure}[htbp]
  \center
  \begin{tikzpicture}[scale=\pattdispscale, baseline={([yshift=-3pt]current bounding box.center)}]
    \def \n {5}
    \foreach \x in {1,...,\n} {
      \draw[gray] (0,\x) -- (\n+1,\x);
      \draw[gray] (\x,0) -- (\x,\n+1);
    }
    \foreach \x in {(1,3),(2,5),(3,1),(4,4),(5,2)} {\fill[black] \x circle (5pt);}
  \end{tikzpicture}
  \caption{The grid representation of the permutation $35142$}
  \label{figure: Gr(35142)}
\end{figure}

\begin{definition}
  The \emph{standardization} of a length $n$ string of unique numbers $s_1 
  \cdots s_n$ is the permutation $\sigma \in \S_n$ which has the same relative 
  ordering as the string, meaning that for every $i, j$, $\sigma(i) < \sigma(j)$ 
  if $s_i < s_j$. We write $\st(s_1 \cdots s_n) = \sigma$.
\end{definition}

Every \emph{index set} $\left\{ i_1, \ldots, i_k \right\} \subseteq \llbracket n 
\rrbracket$ (with $1 \leq i_1 < i_2 < \cdots < i_k \leq n$) \emph{induces} a 
substring $\pi(i_1)\cdots\pi(i_k)$ of length $k$ of $\pi$ and is called an 
\emph{occurrence} of the \emph{pattern} $p = \st(\pi(i_1)\cdots\pi(i_k))$ in 
$\pi$. We say that $\pi$ \emph{contains} $p$ as a \emph{subpermutation} or 
\emph{(classical) pattern} and use the notation $p \preceq \pi$. If there exists 
no index set that induces an occurrence of $p$ we say that $\pi$ \emph{avoids} 
$p$. The set of all patterns contained in $\pi$ is denoted with $\Delta(\pi) = 
\left\{ p \in \S \mid p \preceq \pi \right\}$ and $\Delta(\Pi) = \bigcup_{\pi 
\in \Pi}{\Delta(\pi)}$ for a set of permutations $\Pi$.

% Sometimes we abuse the notation and say that the subset $\left\{ \left( i_j, 
% \pi_j \right) \mid j = 1, \ldots, k \right\}$ of $\Gr{\pi}$ is the occurrence 
% of the pattern $p$ in $\pi$. 

\begin{example}
  The permutation $\pi = 35142$ contains the pattern $q = 213$, highlighted with 
  red circles in Figure~\ref{figure:q in pi}, because $\left\{ 1, 3, 4 \right\}$ 
  induces the substring $\pi(1)\pi(3)\pi(4)= 314$ that has the same relative 
  ordering as $q$ (meaning $\st(314) = 213 = q$). Note that this is the only 
  occurrence of $q$ in $\pi$. 
  \label{example:q in pi}

  \begin{figure}[htbp]
    \center
    \begin{tikzpicture}[scale=\pattdispscale, baseline={([yshift=-3pt]current bounding box.center)}]
      \def \n {5}
      \foreach \x in {1,...,\n} {
        \draw[gray] (0,\x) -- (\n+1,\x);
        \draw[gray] (\x,0) -- (\x,\n+1);
      }
      \foreach \x in {(1,3),(2,5),(3,1),(4,4),(5,2)} {\fill[black] \x circle (5pt);}
      \foreach \x in {(1,3),(3,1),(4,4)} {\draw[thick,red] \x circle (7pt);}
    \end{tikzpicture}
    \caption{An occurrence of $213$ in $35142$ highlighted in red}
    \label{figure:q in pi}
  \end{figure}
\end{example}

The set of all permutations of length $n$ that avoid $\pi$ is denoted by 
$\Avn{\pi}$ and we write $\Av{\pi} = \bigcup\limits_{n=0}^{\infty}\Avn{\pi}$. 
If $\Pi$ is a set of permutations we similarly write $\Av{\Pi} = 
\bigcap\limits_{\pi \in \Pi}\Av{\pi}$, i.e., avoiding a set of permutations 
means avoiding every permutation in the set. Note that $\Av{\emptyset} = \S$.

In contrast to the set of avoiding permutations, we define the 
\emph{containment set} as the complementary, i.e., for a set of permutations 
$\Pi$ we define $\Con{\Pi} = \S_n \setminus \Avn{\pi}$ and write $\Co{\Pi} = 
\bigcup\limits_{n=0}^{\infty}\Con{\Pi}$.

\begin{example}
  The set of all permutations avoiding $21$ is the set of \emph{identity} 
  permutations: \[ \Av{21} = \left\{ \epsilon, 1, 12, 123, \ldots \right\} \]
\end{example}

A \emph{permutation class} is a set $\mc{C}$ of permutations such that 
$\Delta(\mc{C}) \subseteq \mc{C}$. It it well known that $\Av{\Pi}$ is a 
permutation class for any set of permutations $\Pi$. A permutation class is 
\emph{``closed downwards''} as every pattern contained in every permutation in 
$\mc{C}$ is also in $\mc{C}$.


%%%%%%%%%%%%%%%%%%%%%%%%%%%%%%%%%%%%%%%%%%%%%%%%%%%%%%%%%%%%%%%%%%%%%%%%%%%%%%%%
\subsection{Mesh patterns\label{mesh patterns background}}

The concept of \emph{mesh patterns} in permutations was introduced by 
\textcite{branden_mesh_2011} as a pair
\[p = (\pi, R) \text{ with } \pi \in \S_k \text{ and } R \subseteq \llbracket 
0, k \rrbracket \times \llbracket 0, k \rrbracket \] where $\llbracket 0, k 
\rrbracket = \{0, 1, \ldots, k\}$ and $R$ is a set of Cartesian coordinates 
$\boks{i, j}$ denoting the lower left corners of the squares in the grid 
representation of $\pi$ which are \emph{shaded}. An example with $p = (\pi, R)$ 
where $\pi = 213$ and $R = \left\{ \boks{1,2}, \boks{1,3}, \boks{2,2} \right\}$ 
is shown in Figure~\ref{figure: Gr(213, shading)}.

\begin{figure}[htbp]
  \center
  \mpattern{scale=\patttablescale}{ 3 }{ 1/2, 2/1, 3/3 }{ 1/2, 1/3, 2/2 }
  \caption{The mesh pattern $p = \left(213, \left\{ \boks{1,2}, \boks{1,3}, \boks{2,2} \right\} \right)$}
  \label{figure: Gr(213, shading)}
\end{figure}

\begin{definition}[\citeauthor{branden_mesh_2011}]
  An occurrence of $p$ in $\sigma \in \S_n$ is a subset $\omega$ of 
  $\Gr{\sigma}$ such that there are order-preserving injections $\alpha, \beta 
  \colon \llbracket k \rrbracket \mapsto \llbracket n \rrbracket$ satisfying the 
  following two conditions. Firstly, $\omega$ is an occurrence of $\pi$ in the 
  classical sense, that is

    i. $\omega = \left\{ \left( \alpha(i), \beta(j) \right) \mid (i, j) \in \Gr{\pi} \right\}$

  \noindent
  Define $R_{ij} = \left[ \left( \alpha(i) + 1, \alpha(i + 1) - 1 \right) 
  \right] \times \left[ \left( \beta(j) + 1, \beta(j + 1) - 1 \right) \right]$ 
  or $i,j \in \llbracket 0, k \rrbracket$ where $\alpha(0) = \beta(0) = 0$ and 
  $\alpha(k + 1) = \beta(k + 1) = n + 1$. Then the second condition is

    ii. if $\boks{i, j} \in R$ then $R_{ij} \cap \Gr{\sigma} = \emptyset$.

  \noindent
  We call $R_{ij}$ the \emph{region corresponding to} $\boks{i,j}$ and $\pi$ 
  the \emph{underlying classical pattern} of $p$.
\end{definition}

Note that we defined an occurrence of a pattern in a classical permutation to be 
the subsequence of the permutation matching the pattern, but Brändén and 
Claesson defined an occurrence of a pattern in a mesh pattern to be the subset 
of the grid representation matching the pattern. For the sake of simplicity in 
what is to come we will keep with the original author's definition of 
occurrences.

\begin{example}
  Recall Example~\ref{example:q in pi} where we saw that $\pi = 35142$ contains 
  $q = 213$ as a pattern with only one occurrence. There is only one occurrence 
  of $q$ in $\sigma = 53241$ as well, namely $324$. Figure~\ref{figure:p not in 
  pi and p in sigma} shows us two things. One the left side, we see that $p = 
  \left( 213, \left\{ \boks{1,2}, \boks{1,3}, \boks{2,2} \right\} \right)$ is 
  \emph{not} contained in $\pi$, because the point $(2,5) \in \Gr{\pi}$ lies 
  inside $R_{13}$. On the right side we see that $\sigma$ contains $p$, since 
  for the only occurrence of $q$ in $\sigma$ the shaded area does not contain 
  any of the points of $\Gr{\sigma}$. 

  \begin{figure}[htbp]
    \center
    \begin{tikzpicture}[scale=\pattdispscale, baseline={([yshift=-3pt]current bounding box.center)}]
      \useasboundingbox (0.0,-0.1) rectangle (5+1.4,5+1.1);
      
      \foreach [count=\x] \y in {3, 5, 1, 4, 2}
          \filldraw (\x,\y) circle (4pt);
      
      \foreach \x/\y in {1/3, 3/1, 4/4}
          \draw[thick,red] (\x,\y) circle (7pt);

      \foreach \xa/\ya/\xb/\yb in {1/4/3/6, 1/3/3/4, 3/3/4/4}
          \draw[line width=1pt] (\xa+0.1,\ya+0.1) rectangle (\xb-0.1,\yb-0.1);

      \foreach \x/\y in {1/3, 1/4, 1/5, 2/3, 2/4, 2/5, 3/3}
          \fill[pattern color = black!65, pattern=north east lines] (\x,\y) rectangle +(1,1);
      
      \draw[very thin] (1,0.01) -- (1,5.99);
      \draw[very thin] (3,0.01) -- (3,5.99);
      \draw[very thin] (4,0.01) -- (4,5.99);
      \draw[very thin] (0.01,1) -- (5.99,1);
      \draw[very thin] (0.01,3) -- (5.99,3);
      \draw[very thin] (0.01,4) -- (5.99,4);

      \draw[densely dotted, line width=0.6pt] (2,0.01) -- (2,5.99);
      \draw[densely dotted, line width=0.6pt] (5,0.01) -- (5,5.99);
      \draw[densely dotted, line width=0.6pt] (0.01,2) -- (5.99,2);
      \draw[densely dotted, line width=0.6pt] (0.01,5) -- (5.99,5);
    \end{tikzpicture}
    \quad\quad
    \begin{tikzpicture}[scale=\pattdispscale, baseline={([yshift=-3pt]current bounding box.center)}]
      \useasboundingbox (0.0,-0.1) rectangle (5+1.4,5+1.1);
      
      \foreach [count=\x] \y in {5, 3, 2, 4, 1}
          \filldraw (\x,\y) circle (4pt);
      
      \foreach \x/\y in {2/3, 3/2, 4/4}
          \draw[thick,red] (\x,\y) circle (7pt);

      \foreach \xa/\ya/\xb/\yb in {2/4/3/6, 2/3/3/4, 3/3/4/4}
          \draw[line width=1pt] (\xa+0.1,\ya+0.1) rectangle (\xb-0.1,\yb-0.1);

      \foreach \x/\y in {2/3, 2/4, 2/5, 3/3}
          \fill[pattern color = black!65, pattern=north east lines] (\x,\y) rectangle +(1,1);
      
      \draw[very thin] (2,0.01) -- (2,5.99);
      \draw[very thin] (3,0.01) -- (3,5.99);
      \draw[very thin] (4,0.01) -- (4,5.99);
      \draw[very thin] (0.01,2) -- (5.99,2);
      \draw[very thin] (0.01,3) -- (5.99,3);
      \draw[very thin] (0.01,4) -- (5.99,4);

      \draw[densely dotted, line width=0.6pt] (1,0.01) -- (1,5.99);
      \draw[densely dotted, line width=0.6pt] (5,0.01) -- (5,5.99);
      \draw[densely dotted, line width=0.6pt] (0.01,1) -- (5.99,1);
      \draw[densely dotted, line width=0.6pt] (0.01,5) -- (5.99,5);
    \end{tikzpicture}
    \caption{The mesh pattern $p$ from Figure~\ref{figure: Gr(213, shading)} is 
    not contained in $\pi$ (left), but is contained in $\sigma$ (right)}
    \label{figure:p not in pi and p in sigma}
  \end{figure}
\end{example}

Our definitions of $\Av{p}$ and $\Co{p}$ naturally extend to be the set of all 
permutations that avoid and contain $p$, respectively. Similarly $\Av{M} = 
\bigcap\limits_{m \in M}\Av{m}$ and $\Co{M} = \S \setminus \Av{M}$ for a set $M$ 
of mesh patterns.

\emph{Vincular} \cite{babson_generalized_2000}, \emph{covincular} 
\cite{bean_enumerations_2017} and \emph{bivincular} 
\cite{bousquet-melou_2+2-free_2010} permutation patterns are special cases of 
the more general mesh patterns where there are restrictions on the occurrence to 
have adjacent indices, values or both in the permutation. We write them as a 
triple $(\pi, X, Y)$ where $\pi$ is the underlying permutation of length $n$ and 
$X,Y \subseteq \llbracket 0, n \rrbracket$ denote which columns and rows are 
shaded in the mesh pattern. Using our notation, this is equal to the mesh 
pattern \[ (\pi, R) \text{ with } R = \bigcup_{ x \in X }{ \left\{ x \right\} 
\times \llbracket 0, n \rrbracket } \cup \bigcup_{ y \in Y }{ \llbracket 0, n 
\rrbracket \times \left\{ y \right\} } .\] If $X \neq \emptyset$ and $Y = 
\emptyset$ it is a vincular pattern, if $X = \emptyset$ and $Y \neq \emptyset$ 
it is a covincular pattern and if neither $X$ or $Y$ is the empty set it is a 
bivincular pattern. See Figure~\ref{figure: Examples of vincular, covincular 
and bivincular permutation patterns} for examples of a vincular, a covincular 
and a bivincular permutation pattern.

\begin{figure}[htbp]
  \center
  \mpattern{scale=\patttablescale}{ 3 }{ 1/1, 2/2, 3/3 }{1/0, 1/1, 1/2, 1/3, 3/0, 3/1, 3/2, 3/3 }
  \mpattern{scale=\patttablescale}{ 3 }{ 1/2, 2/3, 3/1 }{0/2, 1/2, 2/2, 3/2 }
  \mpattern{scale=\patttablescale}{ 3 }{ 1/3, 2/1, 3/2 }{2/0, 2/1, 2/2, 2/3, 0/1, 1/1, 3/1 }
  \caption{$(123, \{ 1, 3 \}, \emptyset)$, $(231, \emptyset, \{ 2 \})$ and $(312, \{ 2 \}, \{ 1 \})$}
  \label{figure: Examples of vincular, covincular and bivincular permutation patterns}
\end{figure}

So far we have defined the concept of permutations containing permutations and 
mesh patterns. \textcite{tannock_equivalence_2018} defined mesh pattern 
containment of mesh patterns, recalled below.

\begin{definition}[\citeauthor{tannock_equivalence_2018}]
  A mesh pattern $q = (\kappa, T)$ \emph{contains} a mesh pattern $p = 
  (\tau, R)$ as a \emph{subpattern} if $\kappa$ contains $p$ and $\left( 
  \bigcup_{(i,j) \in R}{R_{ij}} \right) \subseteq T$.
\end{definition}

If $p$ does not contain $q$ we say that $p$ \emph{avoids} $q$.


%%%%%%%%%%%%%%%%%%%%%%%%%%%%%%%%%%%%%%%%%%%%%%%%%%%%%%%%%%%%%%%%%%%%%%%%%%%%%%%%
\subsection{Mesh tilings}

\begin{definition}\label{definition:block}
  A \emph{block}\footnote{In the Python source code this is called a 
  \emph{cell}.} $\mf{B}$ is a tuple $(\mc{O}, \mc{R})$ of two sets called 
  \emph{obstructions} and \emph{requirements} of mesh patterns that 
  \emph{generates} the set of permutations \[ \Av{\mc{O}} \cap \Co{\mc{R}} .\]
  The block $\overset{\leftharpoonup}{\mf{B}} = (\mc{R}, \mc{O})$ is called the 
  \emph{flipped} block (of $\mf{B}$).
\end{definition}

We use the term ``block'' interchangeably for the tuple and the set of 
permutations that it generates. 

\begin{example}
  Some notable blocks include the \emph{point block} $(\left\{ 12, 21 \right\}, 
  \left\{ 1 \right\})$, the \emph{empty block} $(\left\{ 1 \right\}, \emptyset)$ 
  and the \emph{free block} $(\emptyset, \emptyset)$ that generate the sets 
  $\left\{ 1 \right\}$, $\left\{ \epsilon \right\}$ and $\S$ respectively.
\end{example}

Before moving on to define mesh tilings we review generalized grid classes 
introduced by \textcite{vatter_small_2011}. Given a permutation $\pi$ of length 
$n$ and sets $X, Y \subseteq \llbracket n \rrbracket$, we denote with $\pi( X 
\times Y )$ the standardization of the subsequence of $\pi$ with indices from 
$X$ and values in $Y$. For example $25134([1, 4] \times [2, 5]) = \st(253) = 
132$.

Now suppose $\mc{M}$ is a $c \times r$ matrix (indexed from left to right and 
bottom to top) whose entries are sets of permutations. An 
\emph{$\mc{M}$-gridding} of the permutation $\pi$ of length $n$ is a pair of 
sequences $1 = c_1 \leq \cdots \leq c_{t+1} = n + 1$ (the column divisions) 
and $1 = r_1 \leq \cdots \leq r_{u+1} = n + 1$ (the row divisions) such that 
$\pi(\left[c_k, c_{k+1} \right) \times \left[r_l, r_{l+1} \right))$ is either 
empty or is a member of $\mc{M}_{k,l}$ for all $k \in \llbracket t \rrbracket$ 
and $l \in \llbracket u \rrbracket$. The \emph{generalized grid class} of 
$\mc{M}$, $\Grid{\mc{M}}$, is the set of all permutations with an 
$\mc{M}$-gridding.

\begin{definition}
  A \emph{mesh tiling} $\mc{M}$ is a $c \times r$ matrix of \emph{blocks} (as in 
  Definition~\ref{definition:block}), representing the set of permutations 
  $\Grid{\mc{M}}$.
\end{definition}

A $1 \times 1$ mesh tiling generates the same set of permutations as the (only) 
block in the mesh tiling.

\begin{example}
  The $2 \times 2$ mesh tiling (depicted in Figure~\ref{figure:mesh tiling of 
  perms starting with one}) with empty blocks in coordinates $(0, 1)$ and 
  $(1, 0)$, a point block at $(0, 0)$ and a free block at $(1, 1)$ generates 
  the set of permutations starting with $1$.

  \begin{figure}[htbp]
    \center
    \strule{\pattdispscale}{2}{2}{(0,0)/\point{2pt}, (1,1)/$\S$}
    \caption{A mesh tiling generating all permutations starting with one}
    \label{figure:mesh tiling of perms starting with one}
  \end{figure}
\end{example}


%%%%%%%%%%%%%%%%%%%%%%%%%%%%%%%%%%%%%%%%%%%%%%%%%%%%%%%%%%%%%%%%%%%%%%%%%%%%%%%%
\section{Implementation\label{mesh tilings implementation}}

With mesh tilings we are seeking to find covers for avoidance sets of 
permutations of the form $\Av{M}$ where $M$ is a set of mesh patterns. Often we 
are interested in the complementary set $\Co{M}$ as well because a cover for it 
might help us understand the structure of the permutation class. In practice, 
therefore, we usually apply \CombCov\ on a single $1 \times 1$ mesh tilings 
hoping to find a cover of the form of one or more mesh tilings of various sizes. 
Our implementation expects the root object to be a $1 \times 1$ mesh tiling but 
it works as well for mesh tilings of larger sizes.

The framework requires us to implement two main functions for our mesh tiling 
object in Python. Given a $c \times r$ mesh tiling $\mc{M}$ these methods are:
\begin{itemize}[(i)]
  \item \texttt{get\_elmnts}: Takes in a parameter $l$ and outputs all elements 
    of length $l$ generated $\mc{M}$ in a consistent order. This is easy to 
    describe mathematically but harder to implement programmatically. We refer 
    to the original implementation by \textcite{bean_permstruct_2017} for 
    further details.
  \item \texttt{get\_subrules}: Generates the subrules $S_i$ of the root object 
    $\mc{M}$. Recall that not all subrules need to be \emph{valid}, but from a 
    performance aspect it preferably does not generate too many invalid rules. 
    Algorithm~\ref{algorithm:mesh tiling subrule generation} shows in 
    pseudo-code how we implemented the subrule generation that yielded the 
    results in Section~\ref{mesh tilings results}. This is the product of 
    several iterations where we edited the algorithm to find more covers while 
    keeping the search space small enough for tolerable running times.
\end{itemize}

\begin{algorithm}[ht]
  \caption{Mesh tiling subrule generation}
  \begin{algorithmic}
  \REQUIRE $1 \times 1$ mesh tiling $\mc{M}$ and positive integers \texttt{max\_columns}, \texttt{max\_rows} and \texttt{max\_active\_blocks}
  \STATE $\mc{C} \leftarrow \mc{M}_{1, 1}$
  \STATE $P \leftarrow \emptyset$
  \FORALL{patterns $p$ in the union of the obstructions and requirements of $\mc{C}$}
    \IF{$p$ is a permutation}
      \STATE append all subpermutations of $p$ to $P$
    \ELSIF{$p$ is a mesh pattern}
      \STATE append all subpatterns of $p$ to $P$
    \ENDIF
  \ENDFOR
  \STATE $C \leftarrow$ the set of $\mc{C}$, $\overset{\leftharpoonup}{\mc{C}}$, the point block and the free block
  \FORALL{patterns $p$ in $P$}
    \STATE append the block $\left( \left\{ p \right\}, \emptyset \right)$ to $C$
  \ENDFOR
  \STATE $M \leftarrow$ the set of the $1 \times 1$ mesh tiling with the empty block
  \FORALL{$i$ such that $1 \leq i \leq \texttt{max\_columns}$}
  \FORALL{$j$ such that $1 \leq j \leq \texttt{max\_rows}$}
  \FORALL{$a$ such that $i * j \leq a \leq \texttt{max\_active\_blocks}$}
    \FORALL{combinations of $a$ blocks (with repetition) from $C$}
      \FORALL{combinations of $a$ squares in an $i \times j$ grid of which there is no column or row with no selected square}
        \STATE append the $i \times j$ mesh tiling with the $a$ blocks placed in the chosen squares to $M$
      \ENDFOR
    \ENDFOR
  \ENDFOR
  \ENDFOR
  \ENDFOR
  \RETURN $M$
  \end{algorithmic}
  \label{algorithm:mesh tiling subrule generation}
\end{algorithm}



%%%%%%%%%%%%%%%%%%%%%%%%%%%%%%%%%%%%%%%%%%%%%%%%%%%%%%%%%%%%%%%%%%%%%%%%%%%%%%%%
\section{Results\label{mesh tilings results}}

Our aim here is twofold. One is to reproduce old results by finding covers for 
sets of permutations avoiding a set of mesh patterns (we will call this 
\emph{avoidance sets} as they are not necessarily permutation classes) and 
compare them to the published results. The other is to try to find covers for 
avoidance sets that are not already known and thus contribute new knowledge to 
the field of enumerative combinatorics.

Just as in Section~\ref{words results} all results presented here were obtained 
using version \texttt{v0.6.3} of \CombCov\ and \texttt{max\_elmnt\_size = 7}.
As these problems demanded more computing power we ran them on the computer 
cluster \emph{Garpur}\footnote{See \url{http://ihpc.is/garpur/} for details.} 
managed by the University of Iceland. The execution times were on the scale of 
hours and days with a hard 2-week time limit after which the execution was 
stopped and deemed unsuccessful.

We inputted 620 different avoidance sets into \CombCov\ along with their 
complementary containment sets for a total of 1240 unique jobs with the 
framework. In each job we tried multiple and increasing values for 
\texttt{max\_columns}, \texttt{max\_rows} and \texttt{max\_active\_blocks}, 
encoded as a triple $(c, r, a)$. Their initial values were set to $(2, 2, 3)$, 
and if a cover was not found we next tried $(3, 3, 3$), after that $(4, 4, 4)$, 
then $(4, 4, 5)$ and lastly $(5, 5, 5)$. We won't summarize all of the results 
here but instead highlight a few interesting covers and discuss their 
significance.


%%%%%%%%%%%%%%%%%%%%%%%%%%%%%%%%%%%%%%%%%%%%%%%%%%%%%%%%%%%%%%%%%%%%%%%%%%%%%%%%
\subsection{Vincular and covincular length 3 mesh patterns\label{Vincular and covincular length 3 mesh patterns results}}

\textcite{claesson_generalized_2001} studied \emph{generalized permutation 
patterns} (which later became called vincular patterns) of length 3 and 
enumerated the permutation sets avoiding one or two such patterns. He used a 
notation of letters and dashes to indicate shaded columns, e.g.,
$a\text{-}bc = \mpattern{scale=\patttextscale}{ 3 }{ 1/1, 2/2, 3/3 }{ 2/0, 2/1, 2/2, 2/3 }$ and
$ca\text{-}b = \mpattern{scale=\patttextscale}{ 3 }{ 1/3, 2/1, 3/2 }{ 1/0, 1/1, 1/2, 1/3 }$.

In Figure~\ref{figure:Cleasson Av(a-bc)} we show the cover which \CombCov\ finds 
for an avoidance set that is enumerated by the \emph{Bell numbers} (OEIS 
sequence number \oeis{A000110}), meaning that $\left| \Avn{\mc{A}} \right|$ is 
the $n$-th Bell number, $B_n$. 

\begin{figure}[htbp]
  \center
    \begin{tabular}{ r c l l }
    $\mc{A} = \Av{ \mpattern{scale=\patttextscale}{ 3 }{ 1/1, 2/2, 3/3 }{ 2/0, 2/1, 2/2, 2/3 } }$ & $=$ & $ 
    \strule{\pattdispscale}{1}{1}{} \mediumsqcup
    \strule{\pattdispscale}{3}{2}{
      (0,1)/$\mc{A}$,
      (1,0)/\point{2pt}, 
      (2,1)/$\mc{B}$
    }$ & $\mc{B} = \Av{ \mpattern{scale=\patttextscale}{ 2 }{ 1/1, 2/2 }{} }$ 
  \end{tabular}
  \caption{\CombCov\ cover of $\Av{a\text{-}bc}$}
  \label{figure:Cleasson Av(a-bc)}
\end{figure}

It's easy to see that the cover indeed generates the permutations that avoid 
$\mpattern{scale=\patttextscale}{ 3 }{ 1/1, 2/2, 3/3 }{ 2/0, 2/1, 2/2, 2/3 }$. 
More interesting is deriving the \emph{exponential generating function (EGF)} 
$F = \sum_{n \geq 0} {a_n \over n!} x^n$ and show that $a_n = B_n$, proving that 
the sequence is enumerated by the Bell numbers.

We start by noting that $\left| \Avn{12} \right| = 1$ for all $n$ so $\mc{B}$ 
in Figure~\ref{figure:Cleasson Av(a-bc)} is enumerated by $(1)_n$ and has EGF 
$\sum_{n \geq 0} {1 \over n!} x^n = e^x$.
This gives us

\begin{align*}
  \text{}     \qquad & F = 1 + \int F e^x dx \\
  \text{i.e.} \qquad & \frac{d}{dx} F = F e^x \\
  \text{i.e.} \qquad & \int \frac{1}{F} dF = \int e^x dx \\
  \text{i.e.} \qquad & \ln(F) = e^x + C \\
  \text{i.e.} \qquad & F = A e^x
\end{align*}

We have shown that $\sum_{n \geq 0} {a_n \over n!} x^n = Ae^x$ for some constant 
$A$. We know that there is only one permutation of length zero in $\mc{A}$ so 
putting $a_0 = 1$ into the equation at $x = 0$ gives us $1 = Ae$ i.e.\@ 
$A = e^{-1}$ so $F = e^{e^x - 1}$ which is indeed the EGF for the Bell numbers.

In Figure~\ref{figure:Cleasson Av(a-bc, a-cb)} we present the cover \CombCov\ 
found for an avoidance set that is enumerated with $I_n$ (\oeis{A000085}), the 
number of \emph{involutions} in $\S_n$.

\begin{figure}[htbp]
  \center
    \begin{tabular}{ r c l l }
    $\mc{A} = \Av{ \mpattern{scale=\patttextscale}{ 3 }{ 1/1, 2/2, 3/3 }{ 2/0, 2/1, 2/2, 2/3 }, \mpattern{scale=\patttextscale}{ 3 }{ 1/1, 2/3, 3/2 }{ 2/0, 2/1, 2/2, 2/3 } }$ & $=$ & $ 
    \strule{\pattdispscale}{1}{1}{} \mediumsqcup
    \strule{\pattdispscale}{2}{2}{
      (0,1)/$\mc{A}$, 
      (1,0)/\point{2pt}
    } \mediumsqcup
    \strule{\pattdispscale}{3}{2}{
      (0,1)/$\mc{A}$, 
      (1,0)/\point{2pt},
      (2,1)/\point{2pt}
    }$ & 
  \end{tabular}
  \caption{\CombCov\ cover of $\Av{a\text{-}bc, a\text{-}cb}$}
  \label{figure:Cleasson Av(a-bc, a-cb)}
\end{figure}

Similarly we can easily see that the cover is indeed correct and derive the 
exponential generating function to prove the enumeration.

\textcite{bean_enumerations_2017} studied the set of permutations simultaneously 
avoiding a vincular and covincular mesh pattern of length 3. An interesting 
result is that \CombCov\ finds the same kind of cover for both $\Av{ 
\mpattern{scale=\patttextscale}{ 3 }{ 1/1, 2/2, 3/3 }{ 1/0, 1/1, 1/2, 1/3 }, 
\mpattern{scale=\patttextscale}{ 3 }{ 1/2, 2/1, 3/3 }{ 0/2, 1/2, 2/2, 3/2 } 
}$ and $\Av{ \mpattern{scale=\patttextscale}{ 3 }{ 1/1, 2/2, 3/3 }{
0/2, 1/2, 2/2, 3/2}, \mpattern{scale=\patttextscale}{ 3 }{ 1/1, 2/3, 3/2 }{} 
}$ which is shown in Figure~\ref{figure:Covincular Motzkin numbers}. These 
avoidance sets are enumerated by the \emph{Motzkin} numbers $M_n$ 
(\oeis{A001006}) and it is easy to see from the cover that the generating 
function fulfills $F(x) = 1 + xF(x) + x^2F(x)^2$ which indeed gives us the 
coefficients $M_n$ to $x^n$.

\begin{figure}[htbp]
  \center
    \begin{tabular}{ r c l l }
    $\mc{A} = \Av{ \mpattern{scale=\patttextscale}{ 3 }{ 1/1, 2/2, 3/3 }{ 1/0, 1/1, 1/2, 1/3 }, \mpattern{scale=\patttextscale}{ 3 }{ 1/2, 2/1, 3/3 }{ 0/2, 1/2, 2/2, 3/2 } }$ & $=$ & $
    \strule{\pattdispscale}{1}{1}{} \mediumsqcup
    \strule{\pattdispscale}{2}{2}{
      (0,1)/$\mc{A}$, 
      (1,0)/\point{2pt}
    } \mediumsqcup
    \strule{\pattdispscale}{4}{4}{
      (0,3)/$\mc{A}$, 
      (1,0)/\point{2pt},
      (2,2)/\point{2pt},
      (3,1)/$\mc{A}$
    } $ & \\
    $\mc{B} = \Av{ \mpattern{scale=\patttextscale}{ 3 }{ 1/1, 2/2, 3/3 }{0/2, 1/2, 2/2, 3/2}, \mpattern{scale=\patttextscale}{ 3 }{ 1/1, 2/3, 3/2 }{} }$ & $=$ & $
    \strule{\pattdispscale}{1}{1}{} \mediumsqcup
    \strule{\pattdispscale}{2}{2}{
      (0,1)/$\mc{B}$, 
      (1,0)/\point{2pt}
    } \mediumsqcup
    \strule{\pattdispscale}{4}{4}{
      (0,3)/$\mc{B}$, 
      (1,0)/\point{2pt},
      (2,2)/\point{2pt},
      (3,1)/$\mc{B}$
    } $ &
  \end{tabular}
  \caption{\CombCov\ cover of avoidance sets that gives rise to the Motzkin numbers}
  \label{figure:Covincular Motzkin numbers}
\end{figure}

Other successful covers in this paper include 
$\Av{ \mpattern{scale=\patttextscale}{ 3 }{ 1/1, 2/3, 3/2 }{ 1/0, 1/1, 1/2, 1/3 }, \mpattern{scale=\patttextscale}{ 3 }{ 1/2, 2/3, 3/1 }{ 0/2, 1/2, 2/2, 3/2 } }$ 
(Figure~\ref{figure:Covincular Av(132_1_e, 231_e_2)}, \oeis{A011782}) and 
$\Av{ \mpattern{scale=\patttextscale}{ 3 }{ 1/1, 2/2, 3/3 }{ 2/0, 2/1, 2/2, 2/3 }, \mpattern{scale=\patttextscale}{ 3 }{ 1/2, 2/3, 3/1 }{ 0/1, 1/1, 2/1, 3/1 } }$ 
(Figure~\ref{figure:Covincular Av(123_2_e, 231_e_1)}, \oeis{A152947}).

\begin{figure}[htbp]
  \center
  \begin{tabular}{ r c l l }
    $\Av{ \mpattern{scale=\patttextscale}{ 3 }{ 1/1, 2/3, 3/2 }{ 1/0, 1/1, 1/2, 1/3 }, \mpattern{scale=\patttextscale}{ 3 }{ 1/2, 2/3, 3/1 }{ 0/2, 1/2, 2/2, 3/2 } }$ & $=$ & $
    \strule{\pattdispscale}{1}{1}{} \mediumsqcup
    \strule{\pattdispscale}{3}{2}{
      (0,1)/$\mc{B}$, 
      (1,0)/\point{2pt},
      (2,1)/$\mc{C}$
    } $ & \begin{tabular}{@{}c@{}}
            $\mc{B} = \Av{ \mpattern{scale=\patttextscale}{ 2 }{ 1/1, 2/2 }{} }$ \\
            $\mc{C} = \Av{ \mpattern{scale=\patttextscale}{ 2 }{ 1/2, 2/1 }{} }$
          \end{tabular}
  \end{tabular}
  \caption{\CombCov\ cover for a pair of vincular and covincular mesh patterns}
  \label{figure:Covincular Av(132_1_e, 231_e_2)}
\end{figure}

\begin{figure}[htbp]
  \center
  \begin{tabular}{ r c l l }
    $\mc{A} = \Av{ \mpattern{scale=\patttextscale}{ 3 }{ 1/1, 2/2, 3/3 }{ 2/0, 2/1, 2/2, 2/3 }, \mpattern{scale=\patttextscale}{ 3 }{ 1/2, 2/3, 3/1 }{ 0/1, 1/1, 2/1, 3/1 } }$ & $=$ & $
    \strule{\pattdispscale}{1}{1}{} \mediumsqcup
    \strule{\pattdispscale}{2}{2}{
      (0,1)/\point{2pt}, 
      (1,0)/$\mc{A}$
    } \mediumsqcup
    \strule{\pattdispscale}{4}{4}{
      (0,1)/$\mc{B}$, 
      (1,0)/\point{2pt},
      (2,3)/$\mc{B}$,
      (3,2)/\point{2pt}
    }$ & $\mc{B} = \Av{ \mpattern{scale=\patttextscale}{ 2 }{ 1/1, 2/2 }{} }$
  \end{tabular}
  \caption{\CombCov\ cover for another pair of vincular and covincular mesh patterns}
  \label{figure:Covincular Av(123_2_e, 231_e_1)}
\end{figure}

We note that $\mc{B} = \left\{ \epsilon, 1, 21, 321, \ldots \right\}$ is the set 
of all decreasing permutations so the generating function for $\Av{ 
\mpattern{scale=\patttextscale}{ 3 }{ 1/1, 2/2, 3/3 }{ 2/0, 2/1, 2/2, 2/3 }, 
\mpattern{scale=\patttextscale}{ 3 }{ 1/2, 2/3, 3/1 }{ 0/1, 1/1, 2/1, 3/1 } 
}$ in Figure~\ref{figure:Covincular Av(123_2_e, 231_e_1)} fulfills \[ F(x) 
= 1 + xF(x) + \frac{x^2}{(1 - x)^2} \] which we can solve to find \[ F(x) = 
\frac{2x^2 - 2x + 1}{(1 - x)^3} \] which gives us the coefficients $1 + {n 
\choose 2}$ to $x^n$.


%%%%%%%%%%%%%%%%%%%%%%%%%%%%%%%%%%%%%%%%%%%%%%%%%%%%%%%%%%%%%%%%%%%%%%%%%%%%%%%%
\subsection{Length 2 mesh patterns\label{Length 2 mesh patterns results}}

\textcite{hilmarsson_wilf-classification_2015} classified all mesh patterns of 
length 2 into \emph{Wilf-equivalence} classes, i.e., the classes of the same 
enumerations. Some notable covers are shown in 
Figures~\ref{figure:WilfShort-1234678}, \ref{figure:WilfShort-234678} and 
\ref{figure:WilfShort-containment}.

\begin{figure}[htbp]
  \center
  \begin{tabular}{ r c l l }
    $\Av{ \mpattern{scale=\patttextscale}{ 2 }{ 1/2, 2/1 }{ 0/0, 0/1, 0/2, 1/0, 1/2, 2/0, 2/1 } }$ & $=$ & $
    \strule{\pattdispscale}{1}{1}{} \mediumsqcup
    \strule{\pattdispscale}{2}{2}{
      (0,0)/\point{2pt}, 
      (1,1)/$\mc{B}$
    } \mediumsqcup
    \strule{\pattdispscale}{2}{2}{
      (0,1)/\point{2pt}, 
      (1,0)/\point{2pt},
      (1,1)/$\S$
    }$ & $\mc{B} = \Av{ \mpattern{scale=\patttextscale}{ 1 }{ 1/1 }{ 0/1, 1/0 } }$ \\
    $\Co{ \mpattern{scale=\patttextscale}{ 2 }{ 1/2, 2/1 }{ 0/0, 0/1, 0/2, 1/0, 1/2, 2/0, 2/1 } }$ & $=$ & $
    \strule{\pattdispscale}{4}{4}{
      (0,0)/\point{2pt}, 
      (1,1)/$\S$,
      (2,2)/\point{2pt},
      (3,3)/$\mc{B}$
    }$ & 
    \end{tabular}
  \caption{Avoidance and containment covers for a length 2 mesh pattern}
  \label{figure:WilfShort-1234678}
\end{figure}

\begin{figure}[htbp]
  \center
  \begin{tabular}{ r c l l }
    $\Av{ \mpattern{scale=\patttextscale}{ 2 }{ 1/2, 2/1 }{ 0/1, 0/2, 1/0, 1/2, 2/0, 2/1 } }$ & $=$ & $
    \strule{\pattdispscale}{1}{1}{
      (0,0)/$\mc{B}$
    } \mediumsqcup
    \strule{\pattdispscale}{3}{3}{
      (0,0)/$\mc{B}$, 
      (1,1)/\point{2pt},
      (2,2)/$\mc{B}$
    }$ & $\mc{B} = \Av{ \mpattern{scale=\patttextscale}{ 1 }{ 1/1 }{ 0/1, 1/0 } }$ \\
    $\Co{ \mpattern{scale=\patttextscale}{ 2 }{ 1/2, 2/1 }{ 0/1, 0/2, 1/0, 1/2, 2/0, 2/1 } }$ & $=$ & $
    \strule{\pattdispscale}{5}{5}{
      (0,0)/$\S$,
      (1,1)/\point{2pt}, 
      (2,2)/$\mc{B}$,
      (3,3)/\point{2pt},
      (4,4)/$\mc{B}$
    }$
  \end{tabular}
  \caption{Avoidance and containment covers for another length 2 mesh pattern}
  \label{figure:WilfShort-234678}
\end{figure}

\begin{figure}[htbp]
  \center
  \begin{tabular}{ r c l l }
    $\Co{ \mpattern{scale=\patttextscale}{ 2 }{ 1/2, 2/1 }{ 0/0, 0/1, 1/0, 1/2, 2/1, 2/2 } }$ & $=$ & $
    \strule{1}{5}{5}{
      (0,4)/$\mc{C}$, 
      (1,1)/\point{2pt},
      (2,2)/$\S$,
      (3,3)/\point{2pt},
      (4,0)/$\S$
    }$ & $\mc{C} = \Av{ \mpattern{scale=\patttextscale}{ 2 }{ 1/2, 2/1 }{ 0/0, 0/1, 1/0, 1/2, 2/1, 2/2 } }$ \\
    $\Co{ \mpattern{scale=\patttextscale}{ 2 }{ 1/2, 2/1 }{ 0/0, 0/1, 1/1, 1/2, 2/0, 2/1 } }$ & $=$ & $
    \strule{1}{5}{4}{
      (0,3)/$\S$, 
      (1,1)/\point{2pt},
      (2,0)/$\mc{D}$,
      (3,2)/\point{2pt},
      (4,3)/$\S$
    }$ & $\mc{D} = \Av{ \mpattern{scale=\patttextscale}{ 2 }{ 1/2, 2/1 }{ 0/0, 0/1, 1/1, 1/2, 2/0, 2/1 } }$ \\
    $\Co{ \mpattern{scale=\patttextscale}{ 2 }{ 1/2, 2/1 }{ 0/0, 0/1, 1/1, 1/2, 2/0, 2/2 } }$ & $=$ & $
    \strule{1}{5}{5}{
      (0,4)/$\S$, 
      (1,1)/\point{2pt},
      (2,0)/$\S$,
      (3,3)/\point{2pt},
      (4,2)/$\S$
    }$ &
  \end{tabular}
  \caption{Containment covers for some length 2 mesh pattern}
  \label{figure:WilfShort-containment}
\end{figure}


%%%%%%%%%%%%%%%%%%%%%%%%%%%%%%%%%%%%%%%%%%%%%%%%%%%%%%%%%%%%%%%%%%%%%%%%%%%%%%%%
\subsection{Bivincular permutation patterns\label{Bivincular permutation patterns results}}

We now focus on some results of bivincular permutation patterns replicated from 
\textcite{parviainen_wilf_2009}. Among them is the cover
\[
  \Av{ \mpattern{scale=\patttextscale}{ 3 }{ 1/1, 2/3, 3/2 }{ 0/3, 1/3, 2/3, 3/3 } } =
  \strule{1}{1}{1}{} \mediumsqcup
  \strule{1}{3}{3}{
    (0,1)/$\S$,
    (1,2)/\point{2pt},
    (2,0)/$\S$
  }
\]
(enumeration \oeis{A003149}) which is especially interesting when comparing it 
to the very well known
\[
  \Av{ \mpattern{scale=\patttextscale}{ 3 }{ 1/1, 2/3, 3/2 }{} } = 
  \strule{1}{1}{1}{} \mediumsqcup
  \strule{1}{3}{3}{
    (0,1)/$\mc{A}$,
    (1,2)/\point{2pt},
    (2,0)/$\mc{A}$
  }
\]
(enumerated by the \emph{Catalan numbers}, \oeis{A000108}) with $\mc{A} = \Av{ 
\mpattern{scale=\patttextscale}{ 3 }{ 1/1, 2/3, 3/2 }{} }$. The 
difference is that $\Av{ \mpattern{scale=\patttextscale}{ 3 }{ 1/1, 2/3, 
3/2 }{ 0/3, 1/3, 2/3, 3/3 } }$ is the set of permutations $\pi$ avoiding 
$231$ in which the $3$ in the occurrence of $231$ is the \emph{topmost point} of 
$\pi$. This is already secured by the point block in the second mesh tiling in 
the cover of $\Av{ \mpattern{scale=\patttextscale}{ 3 }{ 1/1, 2/3, 3/2 }{ 
0/3, 1/3, 2/3, 3/3 } }$ and explains why it includes $\S$ blocks instead 
of the root.

In addition to finding covers for the results presented in the paper, we 
searched for covers for all possible length 3 bivincular mesh patterns, 
discarding symmetries. That made for 163 more avoidance sets to find covers for, 
out of which 79 were successful. Some interesting covers are shown in 
Figure~\ref{figure:WilfBiVincular-avoidance}.

\begin{figure}[htbp]
  \center
  \begin{tabular}{ r c l l }
    $\Av{ \mpattern{scale=\patttextscale}{ 3 }{ 1/1, 2/2, 3/3 }{ 0/0, 0/1, 0/2, 0/3, 1/3, 2/3, 3/3 } }$ & $=$ & $
    \strule{\pattdispscale}{1}{1}{} \mediumsqcup
    \strule{\pattdispscale}{2}{3}{
      (0,1)/\point{2pt},
      (1,0)/$\S$,
      (1,2)/$\mc{B}$
    }$ & $\mc{B} = \Av{ \mpattern{scale=\patttextscale}{ 2 }{ 1/1, 2/2 }{ 0/0, 0/1, 0/2 } }$ \\
    $\Av{ \mpattern{scale=\patttextscale}{ 3 }{ 1/1, 2/2, 3/3 }{ 0/0, 0/1, 0/2, 0/3, 1/0, 1/1, 1/2, 1/3, 2/0, 2/2, 3/0, 3/2 } }$ & $=$ & $ 
    \strule{\pattdispscale}{1}{1}{
      (0,0)/$\mc{C}$
    } \mediumsqcup
    \strule{\pattdispscale}{2}{2}{
      (0,0)/\point{2pt},
      (1,1)/$\mc{D}$
    }$ & \begin{tabular}{@{}c@{}}
            $\mc{C} = \Av{ \mpattern{scale=\patttextscale}{ 1 }{ 1/1 }{ 0/0, 0/1, 1/0 } }$ \\
            $\mc{D} = \Av{ \mpattern{scale=\patttextscale}{ 2 }{ 1/1, 2/2 }{ 0/1, 0/2, 1/1, 2/1 } }$
          \end{tabular} \\
    $\Av{ \mpattern{scale=\patttextscale}{ 3 }{ 1/1, 2/3, 3/2 }{ 0/0, 0/1, 0/2, 0/3, 1/0, 1/1, 1/2, 1/3, 2/0, 2/1, 2/2, 2/3, 3/0, 3/1, 3/3 } }$ & $=$ & $ 
    \strule{\pattdispscale}{1}{1}{
      (0,0)/$\mc{E}$
    } \mediumsqcup
    \strule{\pattdispscale}{3}{3}{
      (0,0)/\point{2pt},
      (1,2)/\point{2pt},
      (2,1)/$\mc{C}$
    }$ & $\mc{E} = \Av{ \mpattern{scale=\patttextscale}{ 2 }{ 1/1, 2/2 }{ 0/0, 0/1, 0/2, 1/0, 1/1, 1/2, 2/0, 2/2 } }$
  \end{tabular}
  \caption{Avoidance covers for some bivincular mesh patterns of length 3}
  \label{figure:WilfBiVincular-avoidance}
\end{figure}


%%%%%%%%%%%%%%%%%%%%%%%%%%%%%%%%%%%%%%%%%%%%%%%%%%%%%%%%%%%%%%%%%%%%%%%%%%%%%%%%
\subsection{Pattern-avoiding Fishburn permutations\label{Pattern-avoiding Fishburn permutations results}}

\textcite{gil_pattern-avoiding_2018} defined \emph{Fishburn permutations} as 
those the avoid the mesh pattern $p = \mpattern{scale=\patttextscale}{ 3 }{ 
1/2, 2/3, 3/1 }{ 0/1, 1/0, 1/1, 2/1, 1/2, 3/1, 1/3 }$ because $\left| \Avn{p} 
\right| = \xi(n)$ are the \emph{Fishburn numbers} (\oeis{A022493}).

They studied the set of Fishburn permutations avoiding a permutation $\pi$ of 
length 3 or 4 (i.e., the set of permutations simultaneously avoiding $p$ and 
$\pi$). For each of the six permutations $\pi$ of length 3, they enumerated all 
sets $\Av{p, \pi}$ and \CombCov\ finds a cover for five of them. Then the 
authors enumerated some of the sets $\Av{p, \pi}$ for $\pi \in \S_4$ but focused 
mostly on proving that $\left| \Avn{p, \pi} \right| = C_n$ (the \emph{Catalan 
numbers}, \oeis{A000108}) for $\pi \in \left\{ 1234, 1243, 1324, 1423, 2134, 
2143, 3124, 3142 \right\}$. We found a cover only for the last $\pi$, indicating 
that \[ \Av{ \mpattern{scale=\patttextscale}{ 3 }{ 1/2, 2/3, 3/1 }{ 0/1, 
1/0, 1/1, 2/1, 1/2, 3/1, 1/3 }, \mpattern{scale=\patttextscale}{ 4 }{ 1/3, 2/1, 
3/4, 4/2 }{} } = \Av{ \mpattern{scale=\patttextscale}{ 3 }{ 1/2, 
2/3, 3/1 }{} } .\] The only other cover we found with a length 4 
permutation $\pi$ is shown in Figure~\ref{figure:Fishburn-1342}.

\begin{figure}[htbp]
  \center
  \begin{tabular}{ r c l l }
    $\Av{ \mpattern{scale=\patttextscale}{ 3 }{ 1/2, 2/3, 3/1 }{ 0/1, 1/0, 1/1, 2/1, 1/2, 3/1, 1/3 } , 
          \mpattern{scale=\patttextscale}{ 4 }{ 1/1, 2/3, 3/4, 4/2 }{} }$ & $=$ & $
    \strule{\pattdispscale}{1}{1}{} \mediumsqcup
    \strule{\pattdispscale}{3}{2}{
      (0,1)/$\mc{B}$,
      (1,0)/\point{2pt},
      (2,1)/$\mc{C}$
    }$ & \begin{tabular}{@{}c@{}}
            $\mc{B} = \Av{ \mpattern{scale=\patttextscale}{ 2 }{ 1/1, 2/2 }{} }$ \\
            $\mc{C} = \Av{ \mpattern{scale=\patttextscale}{ 3 }{ 1/2, 2/3, 3/1 }{} }$
          \end{tabular}
  \end{tabular}
  \caption{Fishburn permutations avoiding $1342$}
  \label{figure:Fishburn-1342}
\end{figure}

In addition to what the authors did we searched for covers of Fishburn 
permutations avoiding two length 4 permutations $\pi_1$ and $\pi_2$. Out of the 
276 avoidance sets $\Av{p, \pi_1, \pi_2}$ we found covers for 10 of them but 
none is interesting enough to show here.


%%%%%%%%%%%%%%%%%%%%%%%%%%%%%%%%%%%%%%%%%%%%%%%%%%%%%%%%%%%%%%%%%%%%%%%%%%%%%%%%
\subsection{Summary}

Out of the 159 results from the papers that we tried to replicate with \CombCov\ 
it found covers for 49 of them. This is summarized in Table~\ref{table:Summary 
of mesh tiling results}, broken down by paper.

\begin{table}[ht]
    \centering
    \begin{tabular}{ l | c | c | c }
        Paper & Section & Results in paper & Covers found \\ \hline
        \cite{claesson_generalized_2001} & \ref{Vincular and covincular length 3 mesh patterns results} & 6 & 5 \\
        \cite{bean_enumerations_2017} & \ref{Vincular and covincular length 3 mesh patterns results} & 40 & 11 \\
        \cite{hilmarsson_wilf-classification_2015} & \ref{Length 2 mesh patterns results} & 65 & 15 \\
        \cite{parviainen_wilf_2009} & \ref{Bivincular permutation patterns results} & 24 & 11 \\
        \cite{gil_pattern-avoiding_2018} & \ref{Pattern-avoiding Fishburn permutations results} & 24 & 7 
    \end{tabular}
    \caption{Summary of replicating mesh pattern covers}
    \label{table:Summary of mesh tiling results}
\end{table}

\chapter{Discussion\label{cha:discussion}}

Just as \Struct\ is a useful tool for finding covers of permutation classes we 
have shown that the generalized version \CombCov\ can find covers for sets of 
other types of combinatorial objects. We defined \emph{words} and \emph{mesh 
patterns} and applied \CombCov\ on both known results and new problems with some 
success.


%%%%%%%%%%%%%%%%%%%%%%%%%%%%%%%%%%%%%%%%%%%%%%%%%%%%%%%%%%%%%%%%%%%%%%%%%%%%%%%%
\section{Conclusion\label{sec:conclusions}}

The framework did not find covers for all of the problems that we tried it with. 
That is not the framework's fault but ours, as the user writing the algorithm on 
generating subrules. We suspect that the algorithm is theoretically capable of 
finding more covers than was presented in this paper but because of slow running 
times and concrete time limits we did not find more covers. \CombCov\ has proved 
its helpfulness by being able to come up with ideas that previously required 
substantial effort on behalf of a human researcher. 


%%%%%%%%%%%%%%%%%%%%%%%%%%%%%%%%%%%%%%%%%%%%%%%%%%%%%%%%%%%%%%%%%%%%%%%%%%%%%%%%
\section{Future work\label{sec:future work}}

Future work includes refining the library with ease-of-use in mind, writing 
documentation and a ``Help get started'' guide. It is our hope that other 
researchers pick it up and apply it to their combinatorial objects of interest 
such as polyominoes, set partitions, alternating sign matrices, ascent sequences 
or something completely different.
%%RUM: "Discussion"

%% ---------------------------------------------------------------
\printbibliography{} %%RUM: "References"

%% If appendices are needed, uncomment the following line
%% and include the appendices in separate files
% \appendix{}%%RUM: "Appendicies (as appropriate)
% \input{code} % as an example, perhaps some of your code

%\backmatter{} % Sections after this don't get numbers
%% We prefer that all elements be numbered

%%%%%%%%%%%%% SHOW INDEX %%%%%%%%%%%%%%%%%%
%% Index, optional.  A good idea on longer documents

% You can put instructions at the beginning of the index:
%\renewcommand{\preindexhook}{%
%  The first page number is usually, but not always,
%  the primary reference to the indexed topic.\vskip\onelineskip}

%% You may have to run "makeindex <FILENAME>" to have it be generated
%% Depending upon which package you chose.
%% 
\clearforchapter{}
\printindex{}%%RUM: Not mentioned

%\backcover{}%%RUM: "Back cover (only Phd)
\end{document}

%% ---------------------------------------------------------------

%%% Local Variables:
%%% mode: latex
%%% TeX-master: t
%%% TeX-engine: xetex
%%% End:
